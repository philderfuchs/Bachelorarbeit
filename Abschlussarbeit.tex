\documentclass[12pt,a4paper,bibliography=totocnumbered,listof=totoc]{scrartcl}
\usepackage[ngerman]{babel}
\usepackage[T1]{fontenc}% wichtig für Trennung von Wörtern mit Umlauten
\usepackage{microtype}% verbesserter Randausgleich
\usepackage[utf8]{inputenc}
\usepackage{amsmath}
\usepackage{amsfonts}
\usepackage{amssymb}
\usepackage{graphicx}
\usepackage{fancyhdr}
\usepackage{tabularx}
\usepackage{geometry}
\usepackage{setspace}
\usepackage[right]{eurosym}
\usepackage[printonlyused]{acronym}
\usepackage{subfig}
\usepackage{floatflt}
\usepackage[usenames,dvipsnames]{color}
\usepackage{colortbl}
\usepackage{paralist}
\usepackage{array}
\usepackage{titlesec}
\usepackage{parskip}
\usepackage[right]{eurosym}
\usepackage{picins}
\usepackage[subfigure,titles]{tocloft}
\usepackage[pdfpagelabels=true]{hyperref}
\usepackage{natbib}
\bibliographystyle{dinat}

\usepackage{listings}
\lstset{basicstyle=\footnotesize, captionpos=b, breaklines=true, showstringspaces=false, tabsize=2, frame=lines, numbers=left, numberstyle=\tiny, xleftmargin=2em, framexleftmargin=2em}
\makeatletter
\def\l@lstlisting#1#2{\@dottedtocline{1}{0em}{1em}{\hspace{1,5em} Lst. #1}{#2}}
\makeatother

\geometry{a4paper, top=27mm, left=30mm, right=30mm, bottom=35mm, headsep=10mm, footskip=12mm}

\hypersetup{unicode=false, pdftoolbar=true, pdfmenubar=true, pdffitwindow=false, pdfstartview={FitH},
	pdftitle={Abschlussarbeit},
	pdfauthor={Philipp Anders},
	pdfsubject={Bachelorarbeit},
	pdfcreator={\LaTeX\ with package \flqq hyperref\frqq},
	pdfproducer={pdfTeX \the\pdftexversion.\pdftexrevision},
	pdfkeywords={Bachelorarebti},
	pdfnewwindow=true,
	colorlinks=true,linkcolor=black,citecolor=black,filecolor=magenta,urlcolor=black}
\pdfinfo{/CreationDate (D:20110620133321)}

\begin{document}

\titlespacing{\section}{0pt}{12pt plus 4pt minus 2pt}{-6pt plus 2pt minus 2pt}

% Kopf- und Fusszeile
\renewcommand{\sectionmark}[1]{\markright{#1}}
\renewcommand{\leftmark}{\rightmark}
\pagestyle{fancy}
\lhead{}
\chead{}
\rhead{\thesection\space\contentsname}
\lfoot{}
\cfoot{}
\rfoot{\thepage}
\renewcommand{\headrulewidth}{0.4pt}
% \renewcommand{\footrulewidth}{0.4pt}

% Vorspann
\renewcommand{\thechapter}{\Roman{chapter}}
\pagenumbering{Roman}
\raggedbottom

% ----------------------------------------------------------------------------------------------------------
% Titelseite
% ----------------------------------------------------------------------------------------------------------
\thispagestyle{empty}
\begin{center}
	\vspace*{1cm}
	\textbf{Hochschule für Technik, Wirtschaft und Kultur Leipzig}\\
	\vspace*{0.3cm}
	Fakultät Informatik, Mathematik und Naturwissenschaften\\
	Bachelorstudiengang Medieninformatik\\
	\vspace*{1cm}
	Bachelorarbeit\\
	zur Erlangung des akademischen Grades\\
	\vspace*{0.3cm}
	\textbf{Bachelor of Science (B.Sc.)}\\
	\Huge
	\vspace*{1cm}
	\textbf{Integration des Tacton Produktkonfigurators in ein Open Source Shopsystem}\\
	
	\vfill
	\normalsize
	\newcolumntype{x}[1]{>{\raggedleft\arraybackslash\hspace{0pt}}p{#1}}
	\begin{tabular}{x{3cm}p{10cm}}
		\rule{0mm}{5ex}\textbf{eingereicht von:} & Philipp Anders\newline Matrikelnummer: 58359 \\ 
		\rule{0mm}{5ex}\textbf{Betreuer:} & Prof. Dr. Michael Frank (HTWK Leipzig)\newline Dipl.-Wirt.-Inf. (FH) Dirk Noack (Lino GmbH)\\ 
		\rule{0mm}{5ex}\textbf{Abgabedatum:} & 29.09.2015 \\ 
	\end{tabular} 
\end{center}
\pagebreak

% ----------------------------------------------------------------------------------------------------------
% Abstract
% ----------------------------------------------------------------------------------------------------------
\setcounter{page}{1}
\onehalfspacing
\titlespacing{\section}{0pt}{12pt plus 4pt minus 2pt}{2pt plus 2pt minus 2pt}
\section*{Kurzfassung}
Das ganze auf Deutsch.

\vspace{-1,2em}
\titlespacing{\section}{0pt}{12pt plus 4pt minus 2pt}{-6pt plus 2pt minus 2pt}
\section*{Abstract}
Das ganze auf Englisch.
\pagebreak

% ----------------------------------------------------------------------------------------------------------
% Verzeichnisse
% ----------------------------------------------------------------------------------------------------------
% TODO Typ vor Nummer
\renewcommand{\cfttabpresnum}{Tab. }
\renewcommand{\cftfigpresnum}{Abb. }
\settowidth{\cfttabnumwidth}{Abb. 10\quad}
\settowidth{\cftfignumwidth}{Abb. 10\quad}

\rhead{Inhaltsverzeichnis}
\setcounter{secnumdepth}{3} 
\setcounter{tocdepth}{3}
\tableofcontents
\pagebreak
\rhead{Verzeichnisse}
\listoffigures
\pagebreak
\listoftables
\pagebreak
\renewcommand{\lstlistlistingname}{Listing-Verzeichnis}
{\labelsep2cm\lstlistoflistings}
\pagebreak

% ----------------------------------------------------------------------------------------------------------
% Abkürzungen
% ----------------------------------------------------------------------------------------------------------
\chapter*{Abkürzungsverzeichnis}
\addcontentsline{toc}{chapter}{Abkürzungsverzeichnis} 
\begin{acronym}[WSDL] % längste Abkürzung steht in eckigen Klammern
	\setlength{\itemsep}{-\parsep} % geringerer Zeilenabstand
	\acro{AJAX}{asynchronous JavaScript and XML}
	\acro{API}{Application Programming Interface}
	\acro{ATO}{Assemble-to-Order}
	\acro{B2B}{Business-to-Business}
	\acro{B2C}{Business-to-Customer}
	\acro{CORS}{Cross-Origin Resource Sharing}
	\acro{CSP}{Constraint Satisfaction Problem}
	\acro{ETO}{Engineer-to-Order}
	\acro{JSON}{JavaScript Object Notation}
	\acro{MC}{Mass Customization}
	\acro{MVC}{Model View Controller}
	\acro{MTO}{Make-to-Order}
	\acro{ORM}{objektrelationale Abbildung}
	\acro{PTO}{Pick-to-Order}
	\acro{RPC}{Remote Procedure Call}
	\acro{SOP}{Same-Origin-Policy}
	\acro{UML}{Unified Modeling Language}
	\acro{URI}{Uniform Resource Identifier}
	\acro{WSDL}{Web Service Description Language}
\end{acronym}
\newpage

% ----------------------------------------------------------------------------------------------------------
% Inhalt
% ----------------------------------------------------------------------------------------------------------
% Abstände Überschrift
\titlespacing{\section}{0pt}{12pt plus 4pt minus 2pt}{0pt plus 2pt minus 2pt}
\titlespacing{\subsection}{0pt}{12pt plus 4pt minus 2pt}{-1pt plus 2pt minus 2pt}
\titlespacing{\subsubsection}{0pt}{12pt plus 4pt minus 2pt}{-1pt plus 2pt minus 2pt}

% Kopfzeile
\renewcommand{\chaptermark}[1]{\markright{#1}}
\renewcommand{\sectionmark}[1]{}
\renewcommand{\subsectionmark}[1]{}
\renewcommand{\subsubsectionmark}[1]{}
\lhead{Kapitel \thechapter}
\rhead{\rightmark}

\onehalfspacing
\renewcommand{\thechapter}{\arabic{chapter}}
\setcounter{chapter}{0}
\pagenumbering{arabic}
\setcounter{page}{0}


\section{Einleitung}

\subsection{Ziel der Arbeit}

\subsection{Aufbau der Arbeit}

\pagebreak

\section{Grundlagen}

\subsection{Bezugsrahmen}
Im Folgenden wird ein Anwendungsrahmen für Produktkonfigurationssysteme (Konfiguratoren) geschaffen. Dazu wird zunächst die marktwirtschaftliche Situation erläutert, welche die Notwendigkeit hybrider Wettbewerbsstrategien begründet. Diese wiederum stellen zu ihrer Umsetzbarkeit Bedingungen an Produktionskonzepte, welche daraufhin vorgestellt werden.

\subsubsection{Ökonomischer Bezug}
\label{subsubsection:oekonomischerBezug}
Neue Wettbewerbsbedingungen und gesteigerte Kundenansprüche haben den Druck auf Industrieunternehmen zur Produktion individualisierter Produkte erhöht \citep{piller98}. Dies erfordert eine stärkere Leistungsdifferenzierung\citep{lutz11}. Die Differenzierung ist ein der von \citeauthor{porter02} beschriebenen \glqq generischen Wettbewerbsstrategien\grqq{}. Weitere sind die Kostenführerschafts- und Fokussierungsstrategie. Die Kostenführerschaft bezieht sich klassischerweise auf Anbieter von Massenproduktion. Durch die Unterbietung der Konkurrenzpreise wird der Marktanteil erhöht. Dem Gegenüber entspricht die Leistungsdifferenzierung dem Verkauf von Produkten mit höherem individuellen Kundennutzen zu höheren Preisen.

\vspace{1em}
\begin{minipage}{\linewidth}
	\centering
	\includegraphics[width=0.7\linewidth]{Abbildungen/unvereinbarkeitshypothese.jpg}
	\captionof{figure}[Unvereinbarkeitshypothese]{Unvereinbarkeitshypothese nach \cite{porter80}, zitiert von \cite{schuh05}}
	\label{fig:unvereinbarkeitshypothese}
\end{minipage}
\vspace{1em}

In diesem Zusammenhang formulierte \citeauthor{porter80} die Unvereinbarkeitshypothese. So sollen Kostenführerschaft und Leistungsdifferenzierung nicht gleichzeitig erreichbar sein. Eine uneindeutige Positionierung führe zu einem \glqq stuck in the middle\grqq{} und damit zur Unwirtschaftlichkeit, wie Abbildung \ref{fig:unvereinbarkeitshypothese} darstellt.

Die Beobachtung der Unternehmensrealität zeichnet ein anderes Bild. Neue Organisatzionsprinzipien, Informationsverarbeitungspotentiale und Produktstrukturierungsansätze ermöglichen einen Kompromiss aus Preis- und Leistungsführerschaft \citep{schuh05}. Das Ergebnis wird als hybride Wettbewerbsstrategien bezeichnet. Eine dieser Strategien ist die sogenannte \ac{MC}.

\ac{MC} ist die \glqq Produktion von Gütern und Leistungen für einen (relativ) großen Absatzmarkt, welche die unterschiedlichen Bedürfnisse jedes einzelnen Nachfragers dieser Produkte treffen, zu Kosten, die ungefähr denen einer massenhaften Fertigung vergleichbarer Standardgüter entsprechen\grqq{} \citep{piller98}. Mit anderen Worten: Preisvorteil (i.d.R. durch Massenfertigung) wird mit Individualisierung (i.d.R. durch Variantenvielfalt) vereint.

\subsubsection{Produktklassifizierung}
 \label{subssubsection:Produktklassifizierung}

\ac{MC} setzt zu deren Umsetzbarkeit gewisse Bedingungen an die Produktionsweisen der abgesetzten Güter. Im Folgenden wird eine Klassifizierung von Produkten in Bezug auf deren Herstellung vorgestellt. Sie werden als Produktionskonzepte bezeichnet  (nach \citealt{schuh06}, zitiert von \citealt{lutz11}):
\begin{compactitem}
	\item \textbf{\ac{PTO}:} Herstellung ohne Kundenauftrag; Lagerhaltung auf Ebene ganzer Produkte; Keine Abhängigkeit dieser Produkte untereinander;\\
	Beispiel: Ein Standardnotebook.
	\item \textbf{\ac{ATO}:} Herstellung ohne Kundeauftrag; Lagerhaltung auf Ebene der Baugruppen/-teilen; Teile mit Abhängigkeiten untereinander;\\
	Beispiel: Ein Notebook, bei welchem auf Kundenwunsch statt des CD-Laufwerks eine zusätzliche Festplatte eingebaut wird. Die Festplatte lag bereits im Lager vor.
	\item \textbf{\ac{MTO}:} Herstellung teilweise erst nach Kundenenauftrag; Lagerhaltung auf Ebene der Baugruppen/-teilen; Produktion oder regelbasierte (parametrisierte) Konstruktion von Komponenten nach Kundenanforderung; Abhängigkeiten zwischen Teilen; keine unendliche Anzahl von Varianten;\\
		Beispiel: Kauf eines Notebooks, wobei der Kunde eine Displaygröße abweichend von den Standarddiagonallängen bestimmen kann. Display und Notebookgehäuse müssen konstruiert/hergestellt und die technischen Standardkomponenten (z.B. Festplatte, Motherboard) eingepasst werden. 
	\item \textbf{\ac{ETO}:} Produkt ist nicht komplett vom Hersteller vorhersehbar; Wenig bis keine Lagerhaltung auf Ebene der Baugruppen/-teilen; Entwicklung und Fertigung von Teilen nach Kundenspezifikation; unendliche Variantenanzahl möglich;\\
	Beispiel: Herstellung eines Notebooks mit Kaffeehalterung.
\end{compactitem}

Die Produktionskonzepte unterscheiden sich hauptsächlich nach dem Kriterium, wann die Produktion der Baugruppen/-teile beginnt - vor oder nach Auftragsspezifikation. Eine Produktion vor Auftragseingang, also ohne Kundenspezifikation, erlaubt Lagerhaltung. Ein hoher Komponentenanteil, der erst nach Auftragseingang hergestellt oder sogar konstruiert werden muss, spricht für eine starke Kundenindividualisierung \citep{lutz11}. Die unterschiedlichen Produktionskonzepte stellen jeweils einen Anwendungsbezug zu Konfiguratoren her, welche im Folgenden vorgestellt werden.

\pagebreak
\subsection{Produktkonfiguration}
\label{subssubsection:Produktkonfiguration}
 
Aus der im vorangegangenen Kapitel vorgestellten \ac{MC} resultiert mehr Produktvariabilität und damit Produktkomplexität. Die Produktkonfiguration (Konfiguration) ist ein Werkzeug zur Beherrschung dieser Komplexität. Sie unterstützt das Finden einer Produktvariante, die auf Kundenanforderungen angepasst und gleichzeitig machbar ist \citep{lutz11}.

\subsubsection{Begriffsüberblick}
\label{subsubsection:begriffsuberblick}
\textbf{Konfiguration} ist eine spezielle Designaktivität, bei der der zu konfigurierende Gegenstand aus Instanzen einer festen Menge wohldefinierter Komponententypen zusammengesetzt wird, welche entsprechend einer Menge von Constraints kombiniert werden können \citep{sabin98}.

Die Einordnung als Designaktivität erlaubt außerdem die Beschreibung der Konfiguration als ein Designtyp. Es werden das \glqq Routine Design\grqq{}, \glqq Innovative Design\grqq{} und \glqq Creative Design\grqq{} unterschieden. Die Konfiguration entspricht dem \glqq Routine Design\grqq{}. Dabei handelt es sich um ein Problem, bei der die Spezifikation der Objekte, deren Eigenschaften sowie kompositionelle Struktur gegeben ist und die Lösung auf Basis einer bekannten Strategie gefunden wird \citep{brown89}. Damit ist \glqq Routine Design\grqq{} die simpelste der drei Formen. Die anderen Designtypen enthalten hingegen Objekte und Objektbeziehungen, die erst während des Designprozesses entwickelt werden., 

Die Schlüsselbegriffe der Definition von \citeauthor{sabin98} sind Komponententypen und Constraints. \textbf{Komponententypen} sind Kombinationselemente, welche durch Attribute charakterisiert werden und eine Menge alternativer (konkreter) Komponenten repräsentieren. Übertragen auf die objektorientierte Programmierung verhalten sich Komponententypen zu Komponenten wie Klassen zu Instanzen. Komponententypen stehen zueinander in Beziehung. Diese kann entweder eine \glqq Teil-Ganzes\grqq{}-Beziehung oder Generalisierung sein \citep{felferning14}. 

\textbf{Constraints} (d.h. Konfigurationsregeln) im engeren Sinne sind Kombinationsrestriktionen \citep{felferning14}. Weitere Arten werden später vorgestellt.

Zur besseren Nachvollziehbarkeit der weiteren Terminologie ist eine Definition des Variantenbegriffs angebracht. DIN 199 beschreibt Varianten als \glqq Gegenstände ähnlicher Form und/oder Funktion mit einem in der Regel hohen Anteil identischer Gruppen oder Teile\grqq{}. Varianten sind also Gegenstandsmengen. Ein Element dieser Menge ist eine konkrete Variantenausprägung. Eine Variantenausprägung unterscheidet sich von einer anderen durch mindestens eine Beziehung oder ein Element \citep{lutz11}.

Die Einheit aus Komponententypen sowie das Wissen um deren Kombinierbarkeit in Form von Constraints wird als \textbf{Konfigurationsmodell} bezeichnet. Es bildet die Menge der korrekten Lösungen ab und definiert so implizit alle Varianten eines Produktes \citep{soininen98}. Dadurch muss nicht jede Variantenausprägung explizit definiert und abgespeichert werden (z.B. in einer Datenbank). Die Anzahl möglicher Kombinationen kann in die Millionen gehen, was die Suche nach einer bestimmten sehr zeitaufwändig machen würde \citep{falkner11}.

Die Einheit aus Konfigurationsmodell und den Kundenanforderungen wird als \textbf{Konfigurationsaufgabe} bezeichnet \citep{felferning14}. Auf dessen Grundlage kann die gewünschte Konfiguration errechnet werden. Demzufolge ist der Begriff Konfiguration überladen: er bezeichnet sowohl den Prozess als auch dessen Ergebnis. Im Folgenden werden daher die Begriffe Konfigurationsprozess sowie Konfigurationslösung verwendet. Der Konfigurationsprozess, der zur Konfigurationslösung führt, wird von einem System durchgeführt. Dieses wird als \textbf{Konfigurator} bezeichnet.

Aus diesem Begriffsüberblick geht hervor, dass die auf dem Konfigurationsmodell basierende Konfigurationsaufgabe der Schlüssel zur Bildung einer kundenspezifischen Variantenausprägung ist. Aus diesem Grunde werden diese Begriffe im Folgenden  genauer erläutert.

\subsubsection{Wissenrepräsentation}
Das Konfigurationswissen beschreibt Wissen, welches über ein konfigurierbares Produkt besteht \citep{soininen98}. Dieses Wissen kann auf unterschiedliche Art und Weise repräsentiert, d.h. dargestellt werden. Die Repräsentation kann zur Definition eines Konfigurationsmodells genutzt werden \citep{felferning14}. Auf konzeptioneller Ebene können die Begriffe Wissensrepräsentation und Konfigurationsmodell äquivalent verwendet werden. Das Konfigurationsmodell bezeichnet jedoch letztendlich das spezifische Format, welches von einem Konfigurator verstanden wird \citep{soininen98}.

Es existieren verschiedene Wissensrepräsentationskonzepte mit unterschiedlicher Ausdruckskraft. Jede hat verschiedene Stärken. Im Folgenden wird ein grafische sowie eine formelle Repräsentationsvariante vorgestellt. Ein Visualisierungskonzept erleichtert den Einstieg und ermöglicht die Bildung einer Vorstellung über die möglichen Varianten eines Produktes. Über die Formalisierung des Konfigurationswissens lässt sich hingegen eine Definition der Konfigurationsaufgabe ableiten.

\textbf{UML-Visualisierung}\\
Von den bestehenden Visualisierungskonzept wird eine UML-basierte Variante besprochen, da diese Sprache in der Informatikdomäne eine besondere Verbreitung aufweist. Im Folgenden wird eine Notebook-Konfiguration eingeführt und von späteren Erklärungen wieder aufgegriffen. Die Modellierung basiert auf dem Arbeitsbeispiel von \citep{felferning14}.

\vspace{1em}
\begin{minipage}{\linewidth}
	\centering
	\includegraphics[width=1\linewidth]{Abbildungen/notebookConfigurationUML.pdf}
	\captionof{figure}[notebookConfigurationUML]{UML-Visualisierung einer Notebook-Konfiguration\footnote{Das 'iTunes-Music-Package' stellt ein Überraschungspaket mit Musik für iTunes dar.}}
	\label{fig:notebookConfigurationUML}
\end{minipage}
\vspace{1em}

\begin{table}[]
\centering
\caption{Constraints des Konfigurationsmodells aus Abbildung \ref{fig:notebookConfigurationUML}}
\label{tab:notebookConfigurationConstraints}
\begin{tabularx}{\textwidth}{|l|X|}
\hline
{\bf Name} & {\bf Beschreibung}\\
\hline
$GC_1$ & Wird das \textbf{iTunes-Music-Package} gewählt, muss auch das Betriebssystem (OS) vom Typ\textbf{OSX Yosemite} gewählt werden. \\
\hline
$GC_2$ & Das OS vom Typ \textbf{OSX Yosemite} und der Arbeitsspeicher (RAM) vom Typ \textbf{Small RAM} können nicht gleichzeitig gewählt werden\\
\hline
$PRC_1$ & Der Preis des Notebooks ist die Summe der \textbf{price}-Attribute der Storage-, Motherboard-, RAM-, CPU- und iTunes-Music-Package-Komponenten\\
\hline
$RESC_1$ & Die Summe der \textbf{needed capacity}-Attribute aller iTunes-Music-Package-Komponenten darf die Summe der \textbf{capacity}-Attribute aller Storage-Komponenten nicht überschreiten\\
\hline
$CRC_1$ & Das OS vom Typ \textbf{OSX Yosemite} benötigt mindestens einen \textbf{core}-Wert der CPU-Komponente von 2\\
\hline
$COMPC_1$ & Das OS vom Typ \textbf{Windows XP} ist inkompatibel mit dem \textbf{size}-Wert der Notebook-Komponente von 13"\\
\hline
\end{tabularx}
\end{table}

Abbildung \ref{fig:notebookConfigurationUML} beschreibt den Strukturteil der Visualisierung. Folgende Spracheinheiten sind enthalten \citep{felferning14}:
\begin{compactitem}
\item \textbf{Komponententypen} sind die dargestellten Entitäten. Sie besitzen einen eindeutigen Namen (z.B. 'Storage') und werden durch eine Menge von Attributen beschrieben (z.B. 'price', 'capacity'). Ein Attribute hat einen Datentyp, welcher eine Konstante, ein Wertebereich (z.B. die Zahlen von 50 bis 100 als mögliche Werte des Attributs 'capacity') oder eine Enumeration (z.B. ['gaming', 'media'] als mögliche Werte des Attributes 'usage') sein kann.
\item \textbf{Generalisierungen} stellen die Verbindungen zwischen einem spezialisierten Subtyp zu einem einem generalisierten Supertyp her. Damit muss der Wertebereich eines Attributs eines Subtypen eine Teilmenge des entsprechenden Attributwertebereichs des Supertypen sein. Angewendet auf das Konfigurationsmodell entsteht so die Unterscheidung zwischen Komponententypen und Komponenten: ein nicht mehr weiter spezialisierter Komponententyp wird als Komponente (unterstrichen dargestellt) bezeichnet. Also sind Generalisierungen die Verbindungen zwischen Komponententypen (z.B. 'Storage') und Komponenten (z.B. 'HD‘). Durch die Zuweisung eines Komponententypen zu einer Komponente entsteht eine Instanz. Diese Zuweisung ist disjunkt und vollständig. Disjunkt bedeutet, dass jede Instanz eines Komponententypen nur genau eine der Komponenten entsprechen kann. Beispiel: Eine Instanz eine 'Storage' kann eine 'HD' oder eine 'SSD' sein, aber nicht beides. Vollständig bedeutet, dass die dargestellten Komponenten alle tatsächlich möglichen Instanzen darstellen (z.B. gibt es für diese Konfiguration keine Komponente 'DVD' als möglichen Storage).
\item \textbf{Assoziationen mit Kardinalitäten} beschreiben die Beziehungen zwischen Komponententypen. Die hier verwendete Variante ist die Komposition. Das bedeutet, dass keine Instanz eines Komponententypen Teil von mehr als einer anderen Instanz sein kann. Kardinalitäten beschreiben Assoziationen noch näher, indem sie sie durch Mengeninformationen ergänzen. Beispiel: Eine Notebook-Instanz besitzt ein oder zwei Storage-Instanzen. Eine Storage-Instanz kann nur Teil einer Notebook-Instanz sein.
\end{compactitem}

Die Darstellung wird ergänzt durch Constraints. Sie gelten zwischen Komponententypen und/oder deren Attribute. Wenn möglich, werden sie direkt im Diagramm dargestellt werden. Anderenfalls werden sie in einer Tabelle aufgelistet (siehe Tabelle \ref{tab:notebookConfigurationConstraints}). Es werden unterschiedliche Constrainttypen unterschieden \citep{felferning14}:

\begin{compactitem}
\item \textbf{Grafische Constraints $GC$} können im Gegensatz zu anderen Constraints direkt im UML-Diagramm dargestellt werden. Ansonsten entsprechen sie einem der folgenden Typen.
\item \textbf{Preisbildungs-Constraints $PRC$} nehmen eine Sonderstellung ein, da sie keinen direkten Einfluss auf die Kombinierbarkeit haben. Stattdessen kann aus der Auswertung dieser Regel eine Preisinformation gewonnen werden. Bei tatsächlichen Konfigurationsanwendungen sind Preisconstraints jedoch meistens nicht Teil des Konfigurationsmodells. Stattdessen wird die Preisbildung durch einen eigenen Mechanismus realisiert.
\item \textbf{Ressourcen-Constraints $RESC$} beschränken die Produktion und den Verbrauch bestimmter Ressourcen. Beispiel:  Jedes 'iTunes-Music-Package' verbraucht 5(MB) Festplattenkapazität. Der verfügbare Speicher wird wiederum durch die Storage-Instanzen bestimmt. Wird nur ein Speichermedium in Form einer 'SSD' gewählt, hat das Notebook 250(MB) Festplattenkapazität. Somit die Obergrenze für 'iTunes-Music-Package' Instanzen gleich 50.
\item \textbf{Abhängigkeits-Constraints $CRC$} beschreiben, unter welchen Voraussetzungen zusätzliche Komponenten Teil der Konfiguration sein müssen.
\item \textbf{Kompatibilitäts-Constraints $COMPC$:} Beschreiben die Kompatibilität  oder Inkompatibilität bestimmter Komponenten.
\end{compactitem}

Die Menge aller Constraints wird auch als Wissensbasis $C_{KB}$ beschrieben. Es gilt:

 $C_{KB} = GC \cup PRC \cup RESC \cup CRC \cup COMPC$

\subsubsection{Konfigurationsaufgabe}
\citet{mittal89} definieren einen Konfigurationsaufgabe wie folgt:
\begin{quote}
(A) a fixed, pre-defined set of components, where a component is described by a set of properties, ports for connecting it to other components, constraints at each port that describe the components that can be connected at that port, and other structural constraints; (B) some description of the desired configuration; and (C) possibly some criteria for making optimal selections.
\end{quote}
(A) ist eine andere Definition für ein Konfigurationsmodell. (B) und (C) sind als Kundenanforderungen zusammenfassbar. Informell entsteht so die im Begriffsüberblick vorgestellte Definition: Die Konfigurationsaufgabe besteht aus dem Konfigurationsmodell sowie den Kundenanforderungen \citep{felferning14}.

Auf formeller Ebene ist eine Konfigurationsaufgabe auch auf Grundlage eines \ac{CSP} beschreibbar. Eine Vorstellung dieser Variante ist sinnvoll, da so eine Definition der Konfigurationslösung abgeleitet werden kann. Das \ac{CSP} ist ein Problem, bei dem für eine gegebene Menge Variablen und deren Wertebereiche unter Berücksichtigung einer Regelmenge versucht wird, eine zulässige Wertekombination ermitteln. Bei einer Konfigurationsaufgabe wird die Regelmenge um die Menge der Kundenanforderungen erweitert \citep{felferning14}.

Eine Konfigurationsaufgabe ist demzufolge ein Tripel $(V, D, C)$, wobei $V = \{v_1, ..., v_n\}$ eine endliche Menge Variablen,  $D  = \{dom(v_1), ..., dom(v_n)\}$ die Menge der Werte der Variablen und $C = C_{KB} \cup REQ$, wobei $C_{KB}$ die oben beschriebene Wissensbasis und $REQ$ die Menge der Kundenanforderungen ist \citep{felferning14}.

\subsubsection{Konfigurationslösung}
Auf Grundlage des \ac{CSP} kann eine Konfigurationslösung formell definiert werden. Es handelt sich dabei um eine Insttanziierung $I = \{v_1 = i_1, ..., v_n = i_n\}$, wobei $i_j$ ein Element aus $dom(v_1)$ ist. I ist vollständig (jede Variable bestitzt einen zugewiesenen Wert) und konsistent (erfüllt alle Constraints) \citep{falkner11}. Eine solche Lösung wird als korrekt bezeichnet \citep{soininen98}.

Eine korrekte Lösung kann durch ein UML-Instanz-Diagramm visualisiert werden.  Abbildung \ref{fig:notebookInstanceUML} zeigt eine korrekte Lösung der Notebook-Konfiguration. Dargestellt wird also eine mögliche Variantenausprägung. Anstatt der Komponententypen sind nur noch konkrete Instanzen enthalten.

\vspace{1em}
\begin{minipage}{\linewidth}
	\centering
	\includegraphics[width=1\linewidth]{Abbildungen/notebookInstanceUML.pdf}
	\captionof{figure}[notebookInstanceUML]{Visualisierung einer Konfigurationslösung als UML-Instanz-Diagramm}
	\label{fig:notebookInstanceUML}
\end{minipage}
\vspace{1em}

Lieferte eine Konfigurationsaufgabe mehr als eine korrekte Lösung, ist das Ergebnis eine Variantenmenge. Eine nicht erfüllbare Konfigurationsaufgabe führt hingegen zu einer leeren Lösungsmenge.

Diese Beschreibung einer Konfigurationslösung suggeriert, dass zu Beginn des Konfigurationsprozesses einmalig die Kundenanforderungen aufgenommen und daraufhin die Lösungsmenge ermittelt wird. Diese Form wird als statische Konfiguration bezeichnet. Demgegenüber erlaubt die interaktive Konfiguration das schrittweise Treffen und Revidieren von Entscheidungen \citep{hadzic04}.

\subsubsection{Konfigurationssysteme}
Der Konfigurator ist das System, welche die Schnittstelle zum Benutzer darstellt und den (interaktiven) Konfigurationsprozess durchführt. Es bekommt die Konfigurationsaufgabe als Eingabe und liefert als Ausgabe die Konfigurationslösung \citep{felferning14}. Konfiguratoren "[...] führen den Abnehmer durch alle Abstimmungsprozesse, die zur Definition des individuellen Produktes nötig sind und prüfen sogleich die Konsistenz sowie Fertigungsfähigkeit der gewünschten Variante" \citep{piller06}.

Nach \citet{piller06} besitzt ein Konfigurator drei Komponenten:
\begin{compactitem}
\item Die \textbf{Konfigurationskomponente} führt den Konfigurationsprozess durch. Sie wird auch als Konfigurationsengine bezeichnet \citep{tactonProductOverview}.
\item Die \textbf{Präsentationskomponente} erstellt eine Konfigurationsdarstellung in zielgruppenspezifischer Form. Daraus lässt sich ableiten, dass sie gleichzeitig als Schnittstelle zur Aufnahme der Kundenanforderungen dient.
\item Die \textbf{Auswertungskomponente} präsentiert der Konfiguration in einer Form, welche eine Interpretation der Variantenausprägung außerhalb des Konfigurators erlaubt. Dies können zum Beispiel Stücklisten, Konstruktionszeichnungen und Arbeitspläne sein.
\end{compactitem}

\textbf{Konfiguratorarten}\\
Konfiguratoren für die Erhebung komplexer Anforderungen technischer Systeme  müssen von Konfiguration für den Einsatz für \ac{MC} unterschieden werden \citep{felferning14}. Erstere sind für den Experteneinsatz gedacht oder dienen nach \citet{piller06} als Vertriebskonfiguratoren der Unterstützung des Verkaufsgespräches.   Letztere werden von Kunden in einer Company-to–Customer Beziehung genutzt und werden auch als Mass Customization Toolkit bezeichnet. Diese sogenannte Selbstkonfiguration ist eine Vorraussetzung für \ac{MC}, indem der zeitkonsumierende Prozess der Erhebung der Kundenbedürfnisse auf die Seite des Kunden verlagert wird \citep{piller06}

Konfiguratoren können bei allen in Abschnit \ref{subssubsection:Produktklassifizierung} genannten Produktionskonzepten zum Einsatz kommen. Je nach Produktionskonzept erfüllen sie für den Anwender eine unterschiedliche Funktion. Bei \ac{PTO} erfüllt der Konfigurator eine Katalogfunktion, indem er den Anwender bei der Auswahl eines fertigen Produktes aus einer Produktpalette unterstützt. Bei \ac{ATO} verhält sich der Konfigurator wie ein Variantengenerator, der den Anwender bei der Auswahl der richtigen Variantenausprägung unterstützt. Wohlgemerkt: der Hersteller hat alle möglichen Variante vordefiniert, sie sind also herstellerspezifisch \citep{schomburg80}. Der Anwendungsfall \ac{MTO} ist ähnlich, jedoch werden Komponenten kundenspezifisch hergestellt oder regelbasiert konstruiert. Es wird von kundenspezifischen Varianten gesprochen \citep{schomburg80}. Bei \ac{ETO} besteht ein erheblicher Neukonstruktionsbedarf. Dies widerspricht der Definition der Konfiguration als Designaktivität aus Abschnitt \ref{subsubsection:begriffsuberblick}
 - die Spezifikation der beteiligten Objekte ist nicht vollständig bekannt. Konfiguration können hier nur einen begrenzt Aus dieser Erläuterung lässt sich Ableiten, dass das Haupteinsatzgebiet von Konfiguratoren im \ac{ATO}/\ac{MTO} Umfeld liegt.

\textbf{Zwischenfazit}\\
In Abschnitt \ref{subssubsection:Produktklassifizierung} wurde dargestellt, wie bestimmte Produktionskonzepte die Herstellung individualisierter Produktvarianten bei gleichzeitiger Lagerfertigung ermöglichen. Produkte werden mit dem Ziel gestaltet, so individuell und auftragsunabhängig wie möglich zu sein. Damit wurde eine der Schlüsselfaktoren für die Ermöglichung der hybriden Wettbewerbsstrategie \ac{MC} erläutert. Diese verbindet die Vorteile effizienter Massenproduktion mit denen
der kundenspezifischen Einzelfertigung\citep{piller98}. \ac{MC} resultiert in Variantenvielfalt und damit in Produktkomplexität. In Abschnitt \ref{subssubsection:Produktkonfiguration}, durch welche Funktionsweise Konfiguratoren zur Beherrschung dieser Komplexität beitragen.


\pagebreak
\subsection{Webservices}

Das \citet	{w3c04} definiert Webservices lose als:

\begin{quote}
\glqq [...] a software system designed to support interoperable machine-to-machine interaction over a network\grqq
\end{quote}

Die Definition schließt die Kommunikation heterogener Systeme ein. \glqq Zwischen Systemen\grqq{} differenziert gleichzeitig klar zur klassischen Verwendung eines Programms, bei der ein (menschlicher) Nutzer mit einem System kommuniziert. \citet{tilkov11} bemerkt, dass Web Service damit sehr weich definiert ist; \glqq nämlich eigentlich gar nicht\grqq{}. Fest steht, dass hier ein Service einen Dienst anbietet, der von einem Clienten über Webtechnologien angesprochen werden kann. Webservices sind demzufolge eine Möglichkeit zur Realisierung von Integrationsszenarien webbasierter Systeme.

\vspace{1em}
\begin{minipage}{\linewidth}
	\centering
	\includegraphics[width=0.7\linewidth]{Abbildungen/clientServerKommunikation.png}
	\captionof{figure}[clientServerKommunikation.png]{Generische Client-Server Kommunikation bei Webservices}
	\label{fig:clientServerKommunikation.png}
\end{minipage}
\vspace{1em}

Abbildung \ref{fig:clientServerKommunikation.png} entspricht im wesentlichen der klassischen Client-Server Kommunikation im Web. Exemplarisch werden XML-Daten übertragen, was die zugrunde liegende Idee der Webservices illustriert: die Übertragung anderer Daten als Webseiten mittels HTTP.

\citet{wilde11} reden von zwei etablierten \glqq Geschmäkern\grqq{} (flavors) in der Webservice Welt: SOAP und REST. Die erste Geschmacksrichtung bedeutet Web Services \glqq auf Basis von SOAP, WSDL und den WS-*Standards - bzw. [...] deren Architektur\grqq{} \citep{tilkov11}. Hier wird also ein XML-basierter Technologiestack beschrieben. REST hingegen ist ein Archikturstil, der 2000 in der Dissertation von \citeauthor{fielding00} vorgestellt wurde. Der Versuch, beide Varianten direkt gegenüberstellen stellen zu wollen, ist ein \glqq [...] klassischer Apfel-Birnenvergleich: ein konkretes XML-Format gegen einen abstrakten Architekturstil\grqq{} \citep{tilkov11}.

Vor einer detaillierteren Diskussion von SOAP und REST wird zur Einordnung eine grundlegende Unterscheidung der Ansätze vorgestellt. Gemeinsam ist beiden, dass HTTP als Transportprotkoll zur Übertragung der Frage (Request) verwendet wird, die vom Server (Response) beantwortert werden soll. HTTP wiederum besteht aus einem Header und einem Entity-Body zur Übertragung von Daten. \citet{richardson07} haben zwei Leitfragen herausgearbeitet, die von den jeweiligen Ansätzen unterschiedlich beantwortet werden: wo in diesem Paket sagt der Client dem Service, mit welchen Daten (Fokusinformation) was (Methodeninformation) gemacht werden soll?

Die Fokusinformation sagt aus, für welche Datenelemente sich der Client interessiert (z.B. ein Artikel eines Onlineshops). Bei REST ist dies der URI zu entnehmen (z.B. http://onlineshop.com/artikel/pc). Bei SOAP steht diese Information in einer XML-Datei welche Entity-Body übertragen wird; die sogenannte Payload. Die Methodeninformation sagt aus, was mit dem identifizierten Datenelement geschehen soll (Bsp:: lege einen neuen PC-Artikel an). Bei REST steht dies im Methodenfeld der HTTP-Headers, bei SOAP wieder im Entity-Body. Daraus lässt sich als grundlegender Unterschied ableiten: SOAP verwendet HTTP nur als Transportprotokoll, REST auch dessen Ausdruckskraft \citep{wilde11}.

\subsubsection{SOAP}
Bei SOAP-Web Services wird ein \ac{RPC} durchgeführt. Dabei handelt es sich um eine generelle Technik zur Realisierung von Systemverteilung. Ein System ruft die Funktion eines Systems aus einem anderen Adressraum auf. SOAP ist ein XML-basiertes Umschlagsformat, welches wiederum die Beschreibung eines Methodenaufrufs in XML-Form enthält. Bei SOAP-Web Services werden also \ac{RPC}s über HTTP getunnelt \citep{wilde11}. Das ist Konvention, aber keine Notwendigkeit: der SOAP-Umschlag ist Transportunabhängig, könnte also auch von anderen Protokollen als HTTP übertragen werden \citep{tilkov11}. Solange es sich bei dem Transportprotokoll um eine Webtechnologie handelt, wird die Webservicedefinition nicht verletzt.

Wie die Beschreibung des \ac{RPC} aussehen muss, definiert die \ac{WSDL}. Jeder SOAP basierte Service stellt eine machinenverarbeitbare \ac{WSDL}-Datei bereit. Darin werden die aufrufbaren Methoden, deren Argumente und Rückgabetypen beschrieben. Außerdem werden Schemata der XML-Dokumente festgehalten, die der Service akzeptiert und versendet \citep{richardson07}.

Es existieren eine Vielzahl von Middleware-Interoperabilitätsstandards, die mit dem \glqq WS-\grqq{} Prefix versehen sind. Diese sind \glqq XML-Aufkleber\grqq{} für den SOAP-Umschlag, die HTTP-Headern entsprechen \citep{richardson07}. Sie erweitern die Ausdrucksmöglichkeit des SOAP-Formats \citep{wilde11}. Beispielsweise erlaubt WS-Security die Berücksichtigung von Sicherheitsaspekten bei der Client-Server Kommunikation. Eine Übersicht der existierenden Standards ist dem Wiki für Webservices \citet{webServiceWiki09} zu entnehmen.

\subsubsection{REST}
\begin{quote}
\glqq Eine Architektur zu definieren bedeutet zu entscheiden, welche Eigenschaften das System haben soll, und eine Reihe von Einschränkungen vorzugeben, mit denen diese Eigenschaften erreicht werden können.\grqq{} \citep{tilkov11}
\end{quote}

Dies ist in der Dissertation von \citeauthor{fielding00} geschehen, in der REST als Architekturstil definiert wird. Ein Architekturstil ist ein stärkerer Abstraktionsgrad als eine Architektur. Beispielsweise ist das Web eine HTTP-Implementierung von REST \citep{tilkov11}. Tatsächlich wurden die Einschränkungen von REST aber dem Web entnommen, indem \citeauthor{fielding00} es post-hoc als lose gekoppeltes, dezentralisiertes Hypermediasystem konzeptualisiert \citep{wilde11} und dann von diesem Konzept abstrahiert hat. Einen Webservice nach dem REST-Architekturstil zu implementieren, bedeutet, es dem Wesen des Webs anzupassen und dessen stärken zu nutzen \citep{tilkov11}.

Entsprechend \citeauthor{tilkov11}s Architekturdefinition werden im Folgenden die Einschränkungen von REST sowie die daraus resultierenden Eigenschaften besprochen.

\textbf{Einschränkungen}\\
Einschränkungen sind - in eigenen Worten - Implementierungskriterien. Während \citeauthor{fielding00} in seiner theoretischen Abhandlung explizit vier solcher Kriterien nennt, basiert die folgende Auflistung auf der praxiserprobten Variante der Sekundärliteratur \citep{wilde11, tilkov11}.

\begin{compactitem}
\item \textbf{Ressourcen mit eindeutiger Identifikation}: \glqq Eine Ressource ist alles, was wichtig genug ist, um als eigenständiges Etwas referenziert zu werden\grqq{} \citep{richardson07}. Identifiziert werden sie im Web durch URIs, die einen globalen Namensraum darstellen.  Es ist hervorzuheben, dass Ressourcen nicht das gleiche sind wie die Datenelemente aus der Persistenzschicht einer Anwendung. Sie befinden sich auf einem anderen Abstraktionsniveau. Beispiel: eine Warenkorbressource kann eine Auflistung von Artikeln sein, welche allerdings nicht einzeln als Ressource ansprechbar sind. \citeauthor{tilkov11} nimmt in diesem Zusammenhang eine Typisierung von Ressourcen vor. Von den sieben verschiedene Ressourcentypen sind folgende im Rahmen der Fragestellung interessant:
\begin{enumerate}[a.]
\item Bei einer \textbf{Projektion} wird die Informationsmenge verringert, indem eine sinnvolle Untermenge der Attribute einer abgerufenen Ressource gebildet wird. Zweck ist die Reduktion der Datenmenge. Beispiel: Weglassen der Beschreibungstexte von Warenkorbartikeln.
\item Die \textbf{Aggregation} ist das Gegenteil. Hier werden Attribute unterschiedlicher Ressourcen zur Reduktion der Anzahl notwendiger Client/Server Interaktionen zusammengefasst. Beispiel: Hinzufügen der Versandkosten beim Abruf der Warenkorbartikel.
\item \textbf{Aktivitäten} sind Ressourcen, die sich aus Prozessen ergeben, wie etwa ein Schritt innerhalb einer Verarbeitung. Beispiel: Ein Schritt einer nicht abgeschlossenen Konfiguration.
\end{enumerate}
\item \textbf{Hypermedia} beschreibt das Prinzip verknüpfter Ressourcen. So wird dem Client ermöglicht, neue Ressourcen zu entdecken oder bestimmte Prozesse anzustoßen. Beispiel: Zur einer Bestellbestätigungsressource wird der zugehörige Stornierungslink hinzugefügt.
\item \textbf{Standardmethoden/uniforme Schnittstelle}: Oben wurde beschreiben, dass jede Ressource durch (mindestens) eine ID identifiziert wird. Jede URI unterstützt dabei den gleichen Methodensatz, welche mit den HTTP-Methoden korrespondieren. Das bedeutet - übertragen auf die objektorientierte Programmierung: jedes Objekt implementiert das gleiche Interface. Folgende Teilmenge der neun verfügbaren HTTP-Methoden finden in der Literatur am häufigsten Erwähnung:
\begin{enumerate}[a.]
\item \textbf{GET:} Das Abholen einer Ressource.
\item \textbf{PUT:} Das Anlegen oder Aktualisieren einer Ressource. Je nachdem, ob unter dieser URI bereits eine Ressource existiert.
\item \textbf{POST:} Bedeutet im engeren Sinne das Anlegen einer Ressource unter einer URI, die vom Service bestimmt. Im weiteren Sinne kann durch Post ein Prozess angestoßen werden.
\item \textbf{Delete:} Das Löschen einer Ressource.
\end{enumerate}

\vspace{1em}
\begin{minipage}{\linewidth}
	\centering
	\includegraphics[width=0.7\linewidth]{Abbildungen/restMethoden.jpg}
	\captionof{figure}[restMethoden]{HTTP-Methoden und ihre Eigenschaften
	\footnote{Relevante Attribute im Rahmen der Fragestellung: \glqq sicher\grqq{} bedeutet Nebenwirkungsfrei, d.h. kein Ressourcenzustand ändert sich durch diese Methode. \glqq Idempotent\grqq{} bedeutet, dass das Resultat der Methode bei Mehrfachausführung das gleiche ist. \glqq Identifizierbare Ressource\grqq{} bedeutet, dass die URL garantiert eine Ressource identifiziert.}
	(Quelle: \citet{tilkov11})}
	\label{fig:restMethoden}
\end{minipage}
\vspace{1em}

Abbildung \ref{fig:restMethoden} fast die Eigenschaften der Methoden aus der HTTP-Spezifikation 1.1 zusammen. Die Implementierung einer Methode muss dem erwarteten Verhalten aus dieser Spezifikation entsprechen. Die Praxis zeigt, dass nur die Methoden unterstützt werden, die für die jeweilige Ressource sinnvoll sind. Abbildung \ref{fig:restMethoden} macht außerdem klar, dass es für POST keinerlei Garantien gibt. Da nicht eindeutig ist, ob über POST eine Ressource erstellt oder ein Prozess angestoßen wird, sehen \citeauthor{richardson07} hierin eine Verletzung der uniformen Schnittstelle. Dies bedeutet in der Praxis: was bei einem Post passiert, ist nicht der HTTP-Spezifikation, sondern der API-Beschreibung des Webservice zu entnehmen.
\item \textbf{Ressourcen und Repräsentationen:} Beschreibt die Darstellungen einer Ressource in einem definierten Format. Der Client bekommt nie die Ressource selbst, sondern nur eine Repräsentation derer zu sehen. In der Praxis wird meist eine  serialisierte Variante eines Objektes als JSON zur Verfügung gestellt. Beispiel: Bereitstellung einer Bestellbestätigung als PDF und HTML.
\item \textbf{Statuslose Kommunikation} bedeutet die Nichtexistenz eines serverseitig abgelegten, transienten, clientspezifischen Status über die Dauer eines Requests hinweg. Der Service benötigt also nie Kontextinformationen zur Bearbeitung eines Requests. Beispiel: Ein Warenkorb wird nicht in einem Sessionobjekt, sondern als persistentes Datenelement gehalten.
\end{compactitem}

Diese Auflistung legt folgende Frage nahe: ist ein Webservice nur dann REST-konform, wenn alle Kriterien erfüllt werden? Was ist mit einem Webservice, der allen Einschränkungen gerecht wird, jedoch Ressourcen nur als JSON ausliefert - ein in der Praxis häufig anzutreffender Fall. Und dennoch ein Verstoß gegen die Forderung nach unterschiedlichen Repräsentationen. Aus diesem Grund existiert das \glqq Richardson Maturity Model\grqq{}, welches die abgestufte Bewertung eines Webservices nach dessen REST-Konformität erlaubt. Es wird im Auswertungsteil vorgestellt und zur Evaluierung der Implementierung genutzt.

\textbf{Eigenschaften}\\
Aus den vorgestellten Kriterien resultieren folgende Eigenschaften \citep{tilkov11},  welche die Vorteile REST-basierter Webservices gegenüber der SOAP-Konkurrenz darstellen \citep{richardson07}:

\begin{compactitem}
\item \textbf{Lose Kopplung:} Beschreibt isolierte Systeme mit größtmöglicher Unabhängigkeit, die über Schnittstellen miteinander kommunizieren. Hierzu tragen die Standardmethoden bei.
\item \textbf{Interoperabilität:} Beschreibt die Möglichkeit der Kommunikation von Systemen unabhängig von deren technischen Implementierung. Dies ergibt sich durch die Festlegung auf Standards. Bei der Anwendung von REST auf Webservices sind dies die Webstandards (z.B. HTTP, URIs).
\item \textbf{Wiederverwendbarkeit}: Jeder Client, der die Schnittstelle eines REST-basierten Service verwenden kann, kann auch jeden anderen beliebigen REST-basierten Service nutzen - vorausgesetzt, das Datenformat wird von beiden Seiten verstanden.
\item \textbf{Performance und Skalierbarkeit}: Im Ideal sollen Webservices schnell antworten, unabhängig von der Anzahl von Anfragen in einem definierten Zeitraum. Dies wird durch Cachebarkeit (siehe HTTP-Methodenspezifikation) und Zustandslosigkeit erreicht. Da der Service keinen clientspezifischen Kontext aufbauen muss, müssen aufeinanderfolgende Requests nicht vom gleichen System beantwortet werden.
\end{compactitem}
\pagebreak

\subsection{eCommerce}

Im Folgenden wird durch die Charakterisierung des Begriffs eCommerce ein Andwendungsrahmen für eShops hergestellt. Deren softwaretechnische Umsetzung wird durch eShop-Systeme realisiert. Durch eine Kategorisierung der Systeme nach Anbieterstrategie wird abschließend die Menge der Open-Source-Lösungen für eine Konfiguratorintegration identifiziert.

\subsubsection{Anwendungsrahmen}
Eshops gehören zur Domäne des elektronischen Handels (eCommerce). Ecommerce ist \glqq die elektronisch unterstützte Abwicklung von Handelsgeschäften auf der Basis der Internet\grqq{} \citep{schwarze02}. Je nachdem, welche Marktpartner an dem Handelsgeschäft teilnehmen, werden verschiedene Formen des eCommerce unterschieden. Die in Abbildung \ref{fig:eCommerceGrundformen} fett hervorgehobenen Varianten werden von \citet{meier12} als \glqq die zwei Geschäftsoptionen des eCommerce\grqq{} bezeichnet: \ac{B2C} und \ac{B2B}. Bei \ac{B2C} erfolgt der Handel von Produkten und Dienstleistungen zwischen Unternehmen and Endverbraucher, bei \ac{B2B} zwischen Unternehmen.

\vspace{1em}
\begin{minipage}{\linewidth}
	\centering
	\includegraphics[width=0.7\linewidth]{Abbildungen/eCommerceGrundformen.jpg}
	\captionof{figure}[eCommerceGrundformen]{Grundformen des eCommerce nach Marktpartnern (Quelle: \citet{schwarze02})}
	\label{fig:eCommerceGrundformen}
\end{minipage}
\vspace{1em}

Für die Umsetzung von eCommerce existieren unterschiedliche Geschäftsmodelle. \citet{timmers98} nennt 11 verschiedene Formen. eShops sind eine eine davon. Es handelt sich dabei um ein \glqq Geschäftsmodell der Angebotsveröffentlichung, bei dem ein Anbieter seine Waren oder Dienstleistungen über das Web den Nachfragern offeriert\grqq{} \citep{bartelt00}.

Ein eShop bildet den traditionellen Einkaufsvorgang nach: Kunden können mittels einer Katalog- oder Suchfunktion über den Produktbestand navigieren. Produkte können ausgewählt und ausführliche, mit Medien angereicherte Beschreibungen abgerufen werden. Wunschartikel werden einem virtuellen Warenkorb hinzugefügt. Ist die Produktauswahl abgeschlossen, begibt sich der Kunde zur \glqq Kasse\grqq{}, wo die Zahlungsmodalitäten erledigt werden \citep{boles00}. Ein eShop beschreibt das Geschäftsmodell, jedoch noch nicht dessen Umsetzung als Softwaresystem. Diese wird als eShop-System bezeichnet \citep{boles00} und im Folgenden behandelt.

\subsubsection{eShop-Systeme}
\glqq eShop-Systeme sind Software-Systeme, die den Aufbau, die Verwaltung und den Einsatz von eShops unterstützen\grqq{} \citep{boles00}. Abbildung \ref{fig:eShopGrobarchitektur} zeigt die Grobarchitektur eines eShop-Systems nach \citeauthor{meier12}. Darin wird die Unterteilung zwischen Storefront und Backfront deutlich, welche in der Terminologie realer Shopsysteme als Front- und Backend bezeichnet werden \citep[vgl.][]{shopwareDoku}. Das Frontend ist der Interaktionsraum der Kunden, das Backend der adminsitrative Bereich des Shopbeitreibers.

\vspace{1em}
\begin{minipage}{\linewidth}
	\centering
	\includegraphics[width=0.7\linewidth]{Abbildungen/eShopGrobarchitektur.jpg}
	\captionof{figure}[eShopGrobarchitektur]{Grobarchitektur eines eShop-Systems (Quelle: \citet{meier12})}
	\label{fig:eShopGrobarchitektur}
\end{minipage}
\vspace{1em}

Die Hauptaufgaben eines eShop-Systems sehen \citet{boles00} in den Bereichen Merchandising (z.B. Management von hierarchisch strukturierten Produktkatalogen, Beeinflussung des Shopdesigns), Auftragsbearbeitung (z.B. Festlegung der Abarbeitungs-Pipeline, Integration von Bezahlverfahren) und Sonstiges (z.B. die Kopplung mit externen ERP-Systemen). Der konkrete Funktionsumfang hängt vom  gewählten eShop-System ab.

Die Systeme sind nach Strategie der Anbieter kategorisierbar:
\begin{compactitem}
\item \textbf{Open-Source}-Systeme sind kostenlos verfügbar. Sie bieten völlige Gestaltungsfreiheit, aber keinen Herstellersupport. Die Dokumentationen sind schwacher und der Funktionsumfang geringer als bei kostenpflichtigen alternativen. Andererseits existieren Communities, die Unterstützung bieten und die Entwicklung von Erweiterungen vorantreiben \citep{stahl15}. Beim kommerziellen Handel der modularen Erweiterungen auf shopspezifischen Stores liegt auch eine der wesentlichen Erlösquellen der Open-Source Strategie (z.B. der Addon Marketplace der \citeauthor{prestashopAddons} oder das Extensionverzeichnis der \citeauthor{opencartExtensions}).
\item \textbf{Kauf-Lösungen} können kostenpflichtig lizensiert werden (z.B. \citeauthor{shopwarePricing}). Sie bieten Herstellersupport, zusätzliche Dienstleistungen (z.B. Installation des Shops) und einen höheren Funktionsumfang (z. B. Schnittstellen zu verschiedenen Warenwirtschaftssystemen oder Zahlungsdienstleistern) \citep{stahl15}. Die Hersteller bieten verschiedene Editionen mit teilweise erheblichen Preisunterschieden an (Bsp.: die Preisdifferenz der Magento Enterprise Edition zu Enterprise Premium liegt bei über 35.000 \$, \citealp[vgl][]{fwpShop}).
\begin{enumerate}[a.]
\item Im Rahmen eines Dual-License-Modells ist eine Open-Source \textbf{Community Edition} Teil des Editionsspektrums \citep{t3n14} (vgl. das Shopangebot der \citeauthor{magentoShops}, \citeauthor{shopwarePricing} oder \citeauthor{oxidShops}). Durch den offenen Quellcode existiert auch hier der Handel modularer Erweiterungen, von dem auch die kostenpflichtigen Varianten profitieren (vgl. der Plugin Store der \citeauthor{shopwarePluginStore}). Aufgrund der gleichen Codebasis alles Editionen kann zu einer Kauf-Lösung migriert werden, was Flexibilität für  wachsende Shopanforderungen bietet.
\end{enumerate}
\item \textbf{Miet-Shops} entsprechen einer Cloud-Lösung als Software-as-a-Service (z.B. \citeauthor{stratoWebshops}, \citeauthor{shopify15}). Die technische Infrastruktur wird vom Provider zur Verfügung gestellt. Systemwartung, Bereitstellung der Shopsoftware und Hosting werden unter dem Mietpreis abgerechnet. \citet{stahl15} bewertet diese Variante als Einstiegslösung mit geringer Gestaltungsfreiheit.
\item \textbf{Eigenentwicklungen} eignen sich für individuelle Bedürfnisse, wenn die Standardsysteme die Anforderungen nicht mehr erfüllen \citep{stahl15, graf14}.
\end{compactitem}

Aus der Liste sind die (zumindest initial) kostenfreien Varianten ersichtlich: reine Open-Source eShop-Systeme sowie die Community-Editionen der Dual-License Modelle. Eine Anbieterübersicht ist \citet{t3n14} zu entnehmen. Eine Kategorisierung der Systeme nach Anforderungsklassen ist \citet{graf14} zu entnehmen.

\pagebreak
\section{Analyse}
Auseinandersetzung mit den konkreten Technologien. Es geht um eine Integration. Was wird integriert? Der Tacton Produktkonfigurator. Das wird als erstes vorgestellt. Worin soll es integriert werden? In ein Open-Source Shopsystem. Das wird als zweites vorgestellt. 

\subsection{Tacton Stuff}
Erst wird Tacton mit der Produktfamilien, dann das Konfigurationsmodell, dann TCSite vorgestellt. Bei TCSite muss der Technologische Background, die KOnfiguration darin sowei die Erweiterbarkeit erwähnt werden.

\subsubsection{Tacton}
Die Tacton Systems AB (Tacton) wurde 1998 Spin-Off des Schwedischen Instituts für Informatik (SICS) gegründet \citep{tactonProductOverview}. In der Forschungseinrichtung wurde als Resultat der Untersuchungen im Bereich Wissensbasierte Systeme und Künstliche Intelligenz der Tacton Produktkonfigurator entwickelt \citep{tactonAbout}. Dieser interaktive Konfigurator ist die Basis der verschiedenen Produkte der Firma.

Neben einer Konfigurationsumgebung bietet Tacton Lösungen im Bereich Vertriebskonfiguration und Design Automation (Automatisierung der Konstruktion in CAD-Systemen). Im Folgenden werden die für die Aufgabenstellung relevanten Produkte vorgestellt
\begin{compactitem}
\item \textbf{Tacton Studio} ist die Konfigurationsumgebung. Es ermöglicht das Erstellen, Testen und Pflegen von Konfigurationsmodellen \citep{tactonProductOverview}. Das Modell wird letztendlich in einem XML-Format abgespeichert \citep{felferning14}. Dadurch wird die Modellentwicklung von Tacton Studio unterstützt, ist aber nicht von dieser abhängig.
\item \textbf{Tacton Configurator Site Application (TCsite)} ist ein webbasierter Vertriebskonfigurator. Er könnte zwar theoretisch von Endkunden genutzt werden, bildet aber nicht die klassischen Geschäftsabläufe eines eShops nach - stattdessen handelt es sich um eine CPQ-Lösung Configure-Price-Quote. Das bedeutet: der Nutzer wird durch den Konfigurationsprozess geführt, was ist einer Preiskalkulation resultiert, woraus ein Angebot erstellt werden kann. Nutzer sind Vertriebsmitarbeiter, Händler oder Partner - der Anwendungsbereich liegt also im B2B. Der Konfigurationsprozess findet im Zusammenhang mit einer Serveranwendung statt, welcher den Produktkonfigurator (die Konfigurationsengine) beherbergt \citep{tactonProductOverview}.
\end{compactitem}

TCsite ist also webbasiert und beherbergt eine Konfigurationsengine. Grundlegende Idee und Gegenstand der Analyse ist also, ob die Konfiguration als Webservice nutzbar gemacht werden kann. Dafür muss das Zusammenspiel der Webanwendung mit der Konfigurationsengine im C-Teil des CPQ-Prozesses untersucht werden. Ist der Konfigurationsprozess in TCsite verstanden, lässt sich daraus ableiten, wie der Service konzipiert werden muss. Voraussetzung ist jedoch Kenntnis darüber, wie Konfigurationswissen bei Tacton in Form eines Konfigurationsmodells abgebildet wird.

\subsubsection{Konfigurationsmodell}

\subsubsection{TCSite}
\textbf{Konfigurationsprozess}
\textbf{Erweiterbarkeit}
Analyse Konfigurationsprozess
Analyse Erweiterbarkeit

\subsection{Shopsystem}

\pagebreak

\section{Anforderungen}

\subsection{Funktionale Anforderungen}

\subsection{Nichtfunktionale Anforderungen}

\pagebreak

\section{Integrationskonzept}

\pagebreak

\section{Integrationsumsetzung}

\pagebreak

\section{Fazit}

\pagebreak

% ----------------------------------------------------------------------------------------------------------
% Vorlagen
% ----------------------------------------------------------------------------------------------------------
\section{Vorlagen}
Dieses Kapitel enthält Beispiele zum Einfügen von Abbildungen, Tabellen, etc.

\subsection{Bilder}
Zum Einfügen eines Bildes, siehe Abbildung \ref{fig:osgi}, wird die \textit{minipage}-Umgebung genutzt, da die Bilder so gut positioniert werden können.

\vspace{1em}
\begin{minipage}{\linewidth}
	\centering
	\includegraphics[width=0.7\linewidth]{Bilder/layering-osgi.png}
	\captionof{figure}[OSGi Architektur]{OSGi Architektur\footnotemark }
	\label{fig:osgi}
\end{minipage}
\footnotetext{Quelle: \url{http://www.osgi.org/Technology/WhatIsOSGi}}

\subsection{Tabellen}
In diesem Abschnitt wird eine Tabelle (siehe Tabelle \ref{tab:beispiel}) dargestellt.

\vspace{1em}
\begin{table}[!h]
	\centering
	\begin{tabular}{|l|l|l|}
		\hline
		\textbf{Name} & \textbf{Name} & \textbf{Name}\\
		\hline
		1 & 2 & 3\\
		\hline
		4 & 5 & 6\\
		\hline
		7 & 8 & 9\\
		\hline
	\end{tabular}
	\caption{Beispieltabelle}
	\label{tab:beispiel}
\end{table}

\pagebreak
\subsection{Auflistung}
Für Auflistungen wird die \textit{compactitem}-Umgebung genutzt, wodurch der Zeilenabstand zwischen den Punkten verringert wird.

\begin{compactitem}
	\item Nur
	\item ein
	\item Beispiel.
\end{compactitem}

\subsection{Listings}
Zuletzt ein Beispiel für ein Listing, in dem Quellcode eingebunden werden kann, siehe Listing \ref{lst:arduino}.

\vspace{1em}
\begin{lstlisting}[caption=Arduino Beispielprogramm, label=lst:arduino]
int ledPin = 13;
void setup() {
    pinMode(ledPin, OUTPUT);
}
void loop() {
    digitalWrite(ledPin, HIGH);
    delay(500);
    digitalWrite(ledPin, LOW);
    delay(500);
}
\end{lstlisting}

\subsection{Tipps}
Die Quellen befinden sich in der Datei \textit{bibo.bib}. Ein Buch- und eine Online-Quelle sind beispielhaft eingefügt.

Abkürzungen lassen sich natürlich auch nutzen. Weiter oben im Latex-Code findet sich das Verzeichnis.
\pagebreak

% ----------------------------------------------------------------------------------------------------------
% Literatur
% ----------------------------------------------------------------------------------------------------------
\pagenumbering{Roman}
\setcounter{page}{1}

\renewcommand\refname{Quellenverzeichnis}
\lhead{}
\bibliographystyle{myalpha}
\bibliography{bibo}
\pagebreak

% ----------------------------------------------------------------------------------------------------------
% Anhang
% ----------------------------------------------------------------------------------------------------------
\pagenumbering{Roman}
\setcounter{page}{1}
\lhead{Anhang \thesection}

\begin{appendix}
\section*{Anhang}
\phantomsection
\addcontentsline{toc}{section}{Anhang}
\addtocontents{toc}{\vspace{-0.5em}}

\vspace{1em}
\begin{minipage}{\linewidth}
	\centering
	\includegraphics[width=1\linewidth]{Abbildungen/tcsiteAdministration.PNG}
	\captionof{figure}[tcsiteAdministration]{TCsite Administrationsoberfläche}
	\label{app:tcsiteAdministration}
\end{minipage}
\vspace{1em}

\vspace{1em}
\begin{minipage}{\linewidth}
	\centering
	\includegraphics[width=1\linewidth]{Abbildungen/tcSiteConfigurationOptionalGroups.PNG}
	\captionof{figure}[tcSiteConfigurationOptionalGroups]{Darstellung optionaler Groups in TCsite}
	\label{app:tcSiteConfigurationOptionalGroups}
\end{minipage}
\vspace{1em}

\end{appendix}

\chapter*{Selbstständigkeitserklärung}
\vspace{2em}
Ich versichere hiermit an Eides statt, dass ich die vorliegende Bachelorarbeit selbstständig und ohne unzulässige fremde Hilfe erbracht habe. Ich habe keine anderen als die angegebenen Quellen und Hilfsmittel benutzt sowie wörtliche und sinngemäße Zitate kenntlich gemacht. Die Arbeit hat in gleicher oder ähnlicher Form noch keiner Prüfungsbehörde
vorgelegen.

\vspace{4em}
\begin{minipage}{\linewidth}
	\begin{tabular}{p{15em}p{15em}}
		Datum: &  .......................................................\\
		& \centering (Unterschrift)\\
	\end{tabular}
\end{minipage}

\end{document}
