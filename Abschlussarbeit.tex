\documentclass[12pt,a4paper,bibliography=totocnumbered,listof=totoc]{scrartcl}
\usepackage[ngerman]{babel}
\usepackage[T1]{fontenc}% wichtig für Trennung von Wörtern mit Umlauten
\usepackage{microtype}% verbesserter Randausgleich
\usepackage[utf8]{inputenc}
\usepackage{amsmath}
\usepackage{amsfonts}
\usepackage{amssymb}
\usepackage{graphicx}
\usepackage{fancyhdr}
\usepackage{tabularx}
\usepackage{geometry}
\usepackage{setspace}
\usepackage[right]{eurosym}
\usepackage[printonlyused]{acronym}
\usepackage{subfig}
\usepackage{floatflt}
\usepackage[usenames,dvipsnames]{color}
\usepackage{colortbl}
\usepackage{paralist}
\usepackage{array}
\usepackage{titlesec}
\usepackage{parskip}
\usepackage[right]{eurosym}
\usepackage{picins}
\usepackage[subfigure,titles]{tocloft}
\usepackage[pdfpagelabels=true]{hyperref}
\usepackage{chngcntr}
\counterwithin{figure}{section}
\usepackage{natbib}
\bibliographystyle{dinat}

\usepackage{listings}
\lstset{basicstyle=\footnotesize, captionpos=b, breaklines=true, showstringspaces=false, tabsize=2, frame=lines, numbers=left, numberstyle=\tiny, xleftmargin=2em, framexleftmargin=2em}
\makeatletter
\def\l@lstlisting#1#2{\@dottedtocline{1}{0em}{1em}{\hspace{1,5em} Lst. #1}{#2}}
\makeatother

\geometry{a4paper, top=27mm, left=30mm, right=30mm, bottom=35mm, headsep=10mm, footskip=12mm}

\hypersetup{unicode=false, pdftoolbar=true, pdfmenubar=true, pdffitwindow=false, pdfstartview={FitH},
	pdftitle={Abschlussarbeit},
	pdfauthor={Philipp Anders},
	pdfsubject={Bachelorarbeit},
	pdfcreator={\LaTeX\ with package \flqq hyperref\frqq},
	pdfproducer={pdfTeX \the\pdftexversion.\pdftexrevision},
	pdfkeywords={Bachelorarebti},
	pdfnewwindow=true,
	colorlinks=true,linkcolor=black,citecolor=black,filecolor=magenta,urlcolor=black}
\pdfinfo{/CreationDate (D:20110620133321)}

\begin{document}

\titlespacing{\section}{0pt}{12pt plus 4pt minus 2pt}{-6pt plus 2pt minus 2pt}

% Kopf- und Fusszeile
\renewcommand{\sectionmark}[1]{\markright{#1}}
\renewcommand{\leftmark}{\rightmark}
\pagestyle{fancy}
\lhead{}
\chead{}
\rhead{\thesection\space\contentsname}
\lfoot{}
\cfoot{}
\rfoot{\thepage}
\renewcommand{\headrulewidth}{0.4pt}
% \renewcommand{\footrulewidth}{0.4pt}

% Vorspann
\renewcommand{\thechapter}{\Roman{chapter}}
\pagenumbering{Roman}
\raggedbottom

% ----------------------------------------------------------------------------------------------------------
% Titelseite
% ----------------------------------------------------------------------------------------------------------
\thispagestyle{empty}
\begin{center}
	\vspace*{1cm}
	\textbf{Hochschule für Technik, Wirtschaft und Kultur Leipzig}\\
	\vspace*{0.3cm}
	Fakultät Informatik, Mathematik und Naturwissenschaften\\
	Bachelorstudiengang Medieninformatik\\
	\vspace*{1cm}
	Bachelorarbeit\\
	zur Erlangung des akademischen Grades\\
	\vspace*{0.3cm}
	\textbf{Bachelor of Science (B.Sc.)}\\
	\Huge
	\vspace*{1cm}
	\textbf{Integration des Tacton Produktkonfigurators in ein Open Source Shopsystem}\\
	
	\vfill
	\normalsize
	\newcolumntype{x}[1]{>{\raggedleft\arraybackslash\hspace{0pt}}p{#1}}
	\begin{tabular}{x{3cm}p{10cm}}
		\rule{0mm}{5ex}\textbf{eingereicht von:} & Philipp Anders\newline Matrikelnummer: 58359 \\ 
		\rule{0mm}{5ex}\textbf{Betreuer:} & Prof. Dr. Michael Frank (HTWK Leipzig)\newline Dipl.-Wirt.-Inf. (FH) Dirk Noack (Lino GmbH)\\ 
		\rule{0mm}{5ex}\textbf{Abgabedatum:} & 29.09.2015 \\ 
	\end{tabular} 
\end{center}
\pagebreak

% ----------------------------------------------------------------------------------------------------------
% Abstract
% ----------------------------------------------------------------------------------------------------------
\setcounter{page}{1}
\onehalfspacing
\titlespacing{\section}{0pt}{12pt plus 4pt minus 2pt}{2pt plus 2pt minus 2pt}
\section*{Kurzfassung}
Das ganze auf Deutsch.

\vspace{-1,2em}
\titlespacing{\section}{0pt}{12pt plus 4pt minus 2pt}{-6pt plus 2pt minus 2pt}
\section*{Abstract}
Das ganze auf Englisch.
\pagebreak

% ----------------------------------------------------------------------------------------------------------
% Verzeichnisse
% ----------------------------------------------------------------------------------------------------------
% TODO Typ vor Nummer
\renewcommand{\cfttabpresnum}{Tab. }
\renewcommand{\cftfigpresnum}{Abb. }
\settowidth{\cfttabnumwidth}{Abb. 10\quad}
\settowidth{\cftfignumwidth}{Abb. 10\quad}

\rhead{Inhaltsverzeichnis}
\setcounter{secnumdepth}{3} 
\setcounter{tocdepth}{3}
\tableofcontents
\pagebreak
\rhead{Verzeichnisse}
\listoffigures
\pagebreak
\listoftables
\pagebreak
\renewcommand{\lstlistlistingname}{Listing-Verzeichnis}
{\labelsep2cm\lstlistoflistings}
\pagebreak

% ----------------------------------------------------------------------------------------------------------
% Abkürzungen
% ----------------------------------------------------------------------------------------------------------
\chapter*{Abkürzungsverzeichnis}
\addcontentsline{toc}{chapter}{Abkürzungsverzeichnis} 
\begin{acronym}[WSDL] % längste Abkürzung steht in eckigen Klammern
	\setlength{\itemsep}{-\parsep} % geringerer Zeilenabstand
	\acro{AJAX}{asynchronous JavaScript and XML}
	\acro{API}{Application Programming Interface}
	\acro{ATO}{Assemble-to-Order}
	\acro{B2B}{Business-to-Business}
	\acro{B2C}{Business-to-Customer}
	\acro{CORS}{Cross-Origin Resource Sharing}
	\acro{CSP}{Constraint Satisfaction Problem}
	\acro{ETO}{Engineer-to-Order}
	\acro{JSON}{JavaScript Object Notation}
	\acro{MC}{Mass Customization}
	\acro{MVC}{Model View Controller}
	\acro{MTO}{Make-to-Order}
	\acro{ORM}{objektrelationale Abbildung}
	\acro{PTO}{Pick-to-Order}
	\acro{RPC}{Remote Procedure Call}
	\acro{SOP}{Same-Origin-Policy}
	\acro{UML}{Unified Modeling Language}
	\acro{URI}{Uniform Resource Identifier}
	\acro{WSDL}{Web Service Description Language}
\end{acronym}
\newpage

% ----------------------------------------------------------------------------------------------------------
% Inhalt
% ----------------------------------------------------------------------------------------------------------
% Abstände Überschrift
\titlespacing{\section}{0pt}{12pt plus 4pt minus 2pt}{0pt plus 2pt minus 2pt}
\titlespacing{\subsection}{0pt}{12pt plus 4pt minus 2pt}{-1pt plus 2pt minus 2pt}
\titlespacing{\subsubsection}{0pt}{12pt plus 4pt minus 2pt}{-1pt plus 2pt minus 2pt}

% Kopfzeile
\renewcommand{\chaptermark}[1]{\markright{#1}}
\renewcommand{\sectionmark}[1]{}
\renewcommand{\subsectionmark}[1]{}
\renewcommand{\subsubsectionmark}[1]{}
\lhead{Kapitel \thechapter}
\rhead{\rightmark}

\onehalfspacing
\renewcommand{\thechapter}{\arabic{chapter}}
\setcounter{chapter}{0}
\pagenumbering{arabic}
\setcounter{page}{0}


\section{Einleitung}

\subsection{Ziel der Arbeit}

\subsection{Aufbau der Arbeit}

\pagebreak

\section{Grundlagen}

\subsection{Bezugsrahmen}
Im Folgenden wird ein Bezugsrahmen für Produktkonfigurationssysteme (Konfiguratoren) geschaffen. Dazu wird zunächst die marktwirtschaftliche Situation erläutert, welche die Notwendigkeit hybrider Wettbewerbsstrategien begründet. Diese wiederum stellen zu ihrer Umsetzbarkeit Bedingungen an Produktionskonzepte, welche daraufhin vorgestellt werden.

\subsubsection{Ökonomischer Bezug}
\label{subsubsection:oekonomischerBezug}
Neue Wettbewerbsbedingungen und gesteigerte Kundenansprüche haben den Druck auf Industrieunternehmen zur Produktion individualisierter Produkte erhöht \citep{piller98}. Dies erfordert eine stärkere Leistungsdifferenzierung \citep{lutz11}. Diese ist eine der von \citeauthor{porter02} beschriebenen \glqq generischen Wettbewerbsstrategien\grqq{}. Sie entspricht dem Verkauf von Produkten mit höherem individuellen Kundennutzen zu höheren Preisen. Eine andere Wettbewerbsstrategie ist die  Fokussierungsstrategie. Diese bezieht sich klassischerweise auf Anbieter von Massenproduktion. Durch die Unterbietung der Konkurrenzpreise wird der Marktanteil erhöht. 

\vspace{1em}
\begin{minipage}{\linewidth}
	\centering
	\includegraphics[width=0.7\linewidth]{Abbildungen/unvereinbarkeitshypothese.jpg}
	\captionof{figure}[Unvereinbarkeitshypothese]{Unvereinbarkeitshypothese nach \cite{porter80}, zitiert von \cite{schuh05}}
	\label{fig:unvereinbarkeitshypothese}
\end{minipage}
\vspace{1em}

In diesem Zusammenhang formulierte \citeauthor{porter80} die Unvereinbarkeitshypothese. So sollen Kostenführerschaft und Leistungsdifferenzierung nicht gleichzeitig erreichbar sein. Eine uneindeutige Positionierung führe zu einem \glqq stuck in the middle\grqq{} und damit zur Unwirtschaftlichkeit, wie Abbildung \ref{fig:unvereinbarkeitshypothese} darstellt.

Die Beobachtung der Unternehmensrealität zeichnet ein anderes Bild. Neue Organisatzionsprinzipien, Informationsverarbeitungspotentiale und Produktstrukturierungsansätze ermöglichen einen Kompromiss aus Preis- und Leistungsführerschaft \citep{schuh05}. Das Ergebnis wird als hybride Wettbewerbsstrategien bezeichnet. Eine dieser Strategien ist die sogenannte \ac{MC}.

\ac{MC} ist die \glqq Produktion von Gütern und Leistungen für einen (relativ) großen Absatzmarkt, welche die unterschiedlichen Bedürfnisse jedes einzelnen Nachfragers dieser Produkte treffen, zu Kosten, die ungefähr denen einer massenhaften Fertigung vergleichbarer Standardgüter entsprechen\grqq{} \citep{piller98}. Mit anderen Worten: Preisvorteil (i.d.R. durch Massenfertigung) wird mit Individualisierung (i.d.R. durch Variantenvielfalt) vereint.

\subsubsection{Produktklassifizierung}
 \label{subssubsection:Produktklassifizierung}

\ac{MC} stellt zu deren Umsetzbarkeit gewisse Bedingungen an die Produktionsweisen der abgesetzten Güter. Im Folgenden wird eine Klassifizierung von Produkten in Bezug auf deren Herstellung vorgestellt. Sie werden als Produktionskonzepte bezeichnet  (nach \citealt{schuh06}, zitiert von \citealt{lutz11}):
\begin{compactitem}
	\item \textbf{\ac{PTO}:} Herstellung ohne Kundenauftrag; Lagerhaltung auf Ebene ganzer Produkte; Keine Abhängigkeit dieser Produkte untereinander;\\
	Beispiel: Ein Standardnotebook.
	\item \textbf{\ac{ATO}:} Herstellung ohne Kundeauftrag; Lagerhaltung auf Ebene der Baugruppen/-teilen; Teile mit Abhängigkeiten untereinander;\\
	Beispiel: Ein Notebook, bei welchem auf Kundenwunsch statt des CD-Laufwerks eine zusätzliche Festplatte eingebaut wird. Die Festplatte lag bereits im Lager vor.
	\item \textbf{\ac{MTO}:} Herstellung teilweise erst nach Kundenenauftrag; Lagerhaltung auf Ebene der Baugruppen/-teilen; Produktion oder regelbasierte (parametrisierte) Konstruktion von Komponenten nach Kundenanforderung; Abhängigkeiten zwischen Teilen; keine unendliche Anzahl von Varianten;\\
		Beispiel: Kauf eines Notebooks, wobei der Kunde eine Displaygröße abweichend von den Standarddiagonallängen bestimmen kann. Display und Notebookgehäuse müssen konstruiert/hergestellt und die technischen Standardkomponenten (z.B. Festplatte, Motherboard) eingepasst werden. 
	\item \textbf{\ac{ETO}:} Produkt ist nicht komplett vom Hersteller vorhersehbar; Wenig bis keine Lagerhaltung auf Ebene der Baugruppen/-teilen; Entwicklung und Fertigung von Teilen nach Kundenspezifikation; unendliche Variantenanzahl möglich;\\
	Beispiel: Herstellung eines Notebooks mit Kaffeehalterung.
\end{compactitem}

Die Produktionskonzepte unterscheiden sich hauptsächlich nach dem Kriterium, wann die Produktion der Baugruppen/-teile beginnt - vor oder nach Auftragsspezifikation. Eine Produktion vor Auftragseingang, also ohne Kundenspezifikation, erlaubt Lagerhaltung. Ein hoher Komponentenanteil, der erst nach Auftragseingang hergestellt oder sogar konstruiert werden muss, spricht für eine starke Kundenindividualisierung \citep{lutz11}. Die unterschiedlichen Produktionskonzepte haben jeweils einen Anwendungsbezug zu Konfiguratoren, welche im Folgenden vorgestellt werden.

\pagebreak
\subsection{Produktkonfiguration}
\label{subssubsection:Produktkonfiguration}
 
Aus der im vorangegangenen Kapitel vorgestellten \ac{MC} resultiert mehr Produktvariabilität und damit Produktkomplexität. Die Produktkonfiguration (Konfiguration) ist ein Werkzeug zur Beherrschung dieser Komplexität. Sie unterstützt das Finden einer Produktvariante, die auf Kundenanforderungen angepasst und gleichzeitig machbar ist \citep{lutz11}.

\subsubsection{Begriffsüberblick}
\label{subsubsection:begriffsuberblick}
\textbf{Konfiguration} ist eine spezielle Designaktivität, bei der der zu konfigurierende Gegenstand aus Instanzen einer festen Menge wohldefinierter Komponententypen zusammengesetzt wird, welche entsprechend einer Menge von Constraints kombiniert werden können \citep{sabin98}.

Die Einordnung als Designaktivität erlaubt außerdem die Beschreibung der Konfiguration als ein Designtyp. Es werden das \glqq Routine Design\grqq{}, \glqq Innovative Design\grqq{} und \glqq Creative Design\grqq{} unterschieden. Die Konfiguration entspricht dem \glqq Routine Design\grqq{}. Dabei handelt es sich um ein Problem, bei der die Spezifikation der Objekte, deren Eigenschaften sowie kompositionelle Struktur gegeben ist und die Lösung auf Basis einer bekannten Strategie gefunden wird \citep{brown89}. Damit ist \glqq Routine Design\grqq{} die simpelste der drei Formen. Die anderen Designtypen enthalten hingegen Objekte und Objektbeziehungen, die erst während des Designprozesses entwickelt werden.

Die Schlüsselbegriffe der Definition von \citeauthor{sabin98} sind Komponententypen und Constraints. \textbf{Komponententypen} sind Kombinationselemente, welche durch Attribute charakterisiert werden und eine Menge alternativer (konkreter) Komponenten repräsentieren. Übertragen auf die objektorientierte Programmierung verhalten sich Komponententypen zu Komponenten wie Klassen zu Instanzen. Komponententypen stehen zueinander in Beziehung. Diese kann entweder eine \glqq Teil-Ganzes\grqq{}-Beziehung oder Generalisierung sein \citep{felferning14}. 

\textbf{Constraints} (d.h. Konfigurationsregeln) im engeren Sinne sind Kombinationsrestriktionen \citep{felferning14}. Weitere Arten werden später vorgestellt.

Zur besseren Nachvollziehbarkeit der weiteren Terminologie ist eine Definition des Variantenbegriffs angebracht. DIN 199 beschreibt Varianten als \glqq Gegenstände ähnlicher Form und/oder Funktion mit einem in der Regel hohen Anteil identischer Gruppen oder Teile\grqq{}. Varianten sind also Gegenstandsmengen. Ein Element dieser Menge ist eine konkrete Variantenausprägung. Eine Variantenausprägung unterscheidet sich von einer anderen durch mindestens eine Beziehung oder ein Element \citep{lutz11}.

Die Einheit aus Komponententypen sowie das Wissen um deren Kombinierbarkeit in Form von Constraints wird als \textbf{Konfigurationsmodell} bezeichnet. Es bildet die Menge der korrekten Lösungen ab und definiert so implizit alle Varianten eines Produktes \citep{soininen98}. Dadurch muss nicht jede Variantenausprägung explizit definiert und abgespeichert werden (z.B. in einer Datenbank). Die Anzahl möglicher Kombinationen kann in die Millionen gehen, was die Suche nach einer bestimmten sehr zeitaufwändig machen würde \citep{falkner11}.

Die Einheit aus Konfigurationsmodell und den Kundenanforderungen wird als \textbf{Konfigurationsaufgabe} bezeichnet \citep{felferning14}. Auf dessen Grundlage kann die gewünschte Konfiguration errechnet werden. Demzufolge ist der Begriff Konfiguration überladen: er bezeichnet sowohl den Prozess als auch dessen Ergebnis. Im Folgenden werden daher die Begriffe Konfigurationsprozess sowie Konfigurationslösung verwendet. Der Konfigurationsprozess, der zur Konfigurationslösung führt, wird von einem System durchgeführt. Dieses wird als \textbf{Konfigurator} bezeichnet.

Aus diesem Begriffsüberblick geht hervor, dass die auf dem Konfigurationsmodell basierende Konfigurationsaufgabe der Schlüssel zur Bildung einer kundenspezifischen Variantenausprägung ist. Aus diesem Grunde werden diese Begriffe im Folgenden  genauer erläutert.

\subsubsection{Wissenrepräsentation}
\label{subsubsection:wissenrepraesentation}
Das Konfigurationswissen beschreibt Wissen, welches über ein konfigurierbares Produkt besteht \citep{soininen98}. Dieses Wissen kann auf unterschiedliche Art und Weise repräsentiert, d.h. dargestellt werden. Die Repräsentation kann zur Definition eines Konfigurationsmodells genutzt werden \citep{felferning14}. Auf konzeptioneller Ebene können die Begriffe Wissensrepräsentation und Konfigurationsmodell äquivalent verwendet werden. Das Konfigurationsmodell bezeichnet jedoch letztendlich das spezifische Format, welches von einem Konfigurator verstanden wird \citep{soininen98}.

Es existieren verschiedene Wissensrepräsentationskonzepte mit unterschiedlicher Ausdruckskraft. Im Folgenden wird ein grafische sowie eine formelle Repräsentationsvariante vorgestellt. Ein Visualisierungskonzept erleichtert den Einstieg und ermöglicht die Bildung einer Vorstellung über die möglichen Varianten eines Produktes. Über die Formalisierung des Konfigurationswissens lässt sich hingegen eine Definition der Konfigurationsaufgabe ableiten.

\textbf{UML-Visualisierung}\\
Von den bestehenden Visualisierungskonzepten wird eine UML-basierte Variante besprochen, da diese Sprache in der Informatikdomäne eine besondere Verbreitung aufweist. Im Folgenden wird eine Notebook-Konfiguration eingeführt und von späteren Erklärungen wieder aufgegriffen. Die Modellierung basiert auf dem Arbeitsbeispiel von \citep{felferning14}.

\vspace{1em}
\begin{minipage}{\linewidth}
	\centering
	\includegraphics[width=1\linewidth]{Abbildungen/notebookConfigurationUML.pdf}
	\captionof{figure}[notebookConfigurationUML]{UML-Visualisierung einer Notebook-Konfiguration\footnote{Das 'iTunes-Music-Package' stellt ein Überraschungspaket mit Musik für iTunes dar.}}
	\label{fig:notebookConfigurationUML}
\end{minipage}
\vspace{1em}

\begin{table}[]
\centering
\caption{Constraints des Konfigurationsmodells aus Abbildung \ref{fig:notebookConfigurationUML}}
\label{tab:notebookConfigurationConstraints}
\begin{tabularx}{\textwidth}{|l|X|}
\hline
{\bf Name} & {\bf Beschreibung}\\
\hline
$GC_1$ & Wird das \textbf{iTunes-Music-Package} gewählt, muss auch das Betriebssystem (OS) vom Typ\textbf{OSX Yosemite} gewählt werden. \\
\hline
$GC_2$ & Das OS vom Typ \textbf{OSX Yosemite} und der Arbeitsspeicher (RAM) vom Typ \textbf{Small RAM} können nicht gleichzeitig gewählt werden\\
\hline
$PRC_1$ & Der Preis des Notebooks ist die Summe der \textbf{price}-Attribute der Storage-, Motherboard-, RAM-, CPU- und iTunes-Music-Package-Komponenten\\
\hline
$RESC_1$ & Die Summe der \textbf{needed capacity}-Attribute aller iTunes-Music-Package-Komponenten darf die Summe der \textbf{capacity}-Attribute aller Storage-Komponenten nicht überschreiten\\
\hline
$CRC_1$ & Das OS vom Typ \textbf{OSX Yosemite} benötigt mindestens einen \textbf{core}-Wert der CPU-Komponente von 2\\
\hline
$COMPC_1$ & Das OS vom Typ \textbf{Windows XP} ist inkompatibel mit dem \textbf{size}-Wert der Notebook-Komponente von 13"\\
\hline
\end{tabularx}
\end{table}

Abbildung \ref{fig:notebookConfigurationUML} beschreibt den Strukturteil der Visualisierung. Folgende Spracheinheiten sind enthalten \citep{felferning14}:
\begin{compactitem}
\item \textbf{Komponententypen} sind die dargestellten Entitäten. Sie besitzen einen eindeutigen Namen (z.B. 'Storage') und werden durch eine Menge von Attributen beschrieben (z.B. 'price', 'capacity'). Ein Attribute hat einen Datentyp, welcher eine Konstante, ein Wertebereich (z.B. die Zahlen von 50 bis 100 als mögliche Werte des Attributs 'capacity') oder eine Enumeration (z.B. ['gaming', 'media'] als mögliche Werte des Attributes 'usage') sein kann.
\item \textbf{Generalisierungen} stellen die Verbindungen zwischen einem spezialisierten Subtyp zu einem einem generalisierten Supertyp her. Damit muss der Wertebereich eines Attributs eines Subtypen eine Teilmenge des entsprechenden Attributwertebereichs des Supertypen sein. Angewendet auf das Konfigurationsmodell entsteht so die Unterscheidung zwischen Komponententypen und Komponenten: ein nicht mehr weiter spezialisierter Komponententyp wird als Komponente (unterstrichen dargestellt) bezeichnet. Also sind Generalisierungen die Verbindungen zwischen Komponententypen (z.B. 'Storage') und Komponenten (z.B. 'HD‘). Durch die Zuweisung eines Komponententypen zu einer Komponente entsteht eine Instanz. Diese Zuweisung ist disjunkt und vollständig. Disjunkt bedeutet, dass jede Instanz eines Komponententypen nur genau eine der Komponenten entsprechen kann. Beispiel: Eine Instanz eine 'Storage' kann eine 'HD' oder eine 'SSD' sein, aber nicht beides. Vollständig bedeutet, dass die dargestellten Komponenten alle tatsächlich möglichen Instanzen darstellen (z.B. gibt es für diese Konfiguration keine Komponente 'DVD' als möglichen Storage).
\item \textbf{Assoziationen mit Kardinalitäten} beschreiben die Beziehungen zwischen Komponententypen. Die hier verwendete Variante ist die Komposition. Das bedeutet, dass keine Instanz eines Komponententypen Teil von mehr als einer anderen Instanz sein kann. Kardinalitäten beschreiben Assoziationen noch näher, indem sie sie durch Mengeninformationen ergänzen. Beispiel: Eine Notebook-Instanz besitzt ein oder zwei Storage-Instanzen. Eine Storage-Instanz kann nur Teil einer Notebook-Instanz sein.
\end{compactitem}

Die Darstellung wird ergänzt durch Constraints. Sie gelten zwischen Komponententypen und/oder deren Attribute. Wenn möglich, werden sie direkt im Diagramm dargestellt. Anderenfalls werden sie in einer Tabelle aufgelistet (siehe Tabelle \ref{tab:notebookConfigurationConstraints}). Es werden folgende Constrainttypen unterschieden \citep{felferning14}:

\begin{compactitem}
\item \textbf{Grafische Constraints $GC$} können im Gegensatz zu anderen Constraints direkt im UML-Diagramm dargestellt werden. Ansonsten entsprechen sie einem der folgenden Typen.
\item \textbf{Preis-Constraints $PRC$} beziehen sich auf die Preisbildung der Konfiguration. Bei Tatsächliche Konfigurationssoftware sind diese jedoch meistens nicht Teil des Konfigurationsmodells. Stattdessen wird die Preisbildung durch einen eigenen Mechanismus realisiert.
\item \textbf{Ressourcen-Constraints $RESC$} beschränken die Produktion und den Verbrauch bestimmter Ressourcen. Beispiel:  Jedes 'iTunes-Music-Package' verbraucht 5(MB) Festplattenkapazität. Der verfügbare Speicher wird wiederum durch die Storage-Instanzen bestimmt. Wird nur ein Speichermedium in Form einer 'SSD' gewählt, hat das Notebook 250(MB) Festplattenkapazität. Somit die Obergrenze für 'iTunes-Music-Package' Instanzen gleich 50.
\item \textbf{Abhängigkeits-Constraints $CRC$} beschreiben, unter welchen Voraussetzungen zusätzliche Komponenten Teil der Konfiguration sein müssen.
\item \textbf{Kompatibilitäts-Constraints $COMPC$:} Beschreiben die Kompatibilität  oder Inkompatibilität bestimmter Komponenten.
\end{compactitem}

Die Menge aller Constraints wird auch als Wissensbasis $C_{KB}$ beschrieben. Es gilt:

 $C_{KB} = GC \cup PRC \cup RESC \cup CRC \cup COMPC$

\subsubsection{Konfigurationsaufgabe}
\citet{mittal89} definieren einen Konfigurationsaufgabe wie folgt:
\begin{quote}
(A) a fixed, pre-defined set of components, where a component is described by a set of properties, ports for connecting it to other components, constraints at each port that describe the components that can be connected at that port, and other structural constraints; (B) some description of the desired configuration; and (C) possibly some criteria for making optimal selections.
\end{quote}
(A) ist eine andere Definition für ein Konfigurationsmodell. (B) und (C) sind als Kundenanforderungen zusammenfassbar. Informell entsteht so die im Begriffsüberblick vorgestellte Definition: Die Konfigurationsaufgabe besteht aus dem Konfigurationsmodell sowie den Kundenanforderungen \citep{felferning14}.

Auf formeller Ebene ist eine Konfigurationsaufgabe auch auf Grundlage eines \ac{CSP} beschreibbar. Eine Vorstellung dieser Variante ist sinnvoll, da so eine Definition der Konfigurationslösung abgeleitet werden kann. Das \ac{CSP} ist ein Problem, bei dem für eine gegebene Menge Variablen und deren Wertebereiche unter Berücksichtigung einer Regelmenge versucht wird, eine zulässige Wertekombination ermitteln. Bei einer Konfigurationsaufgabe wird die Regelmenge um die Menge der Kundenanforderungen erweitert \citep{felferning14}.

Eine Konfigurationsaufgabe ist demzufolge ein Tripel $(V, D, C)$, wobei $V = \{v_1, ..., v_n\}$ eine endliche Menge Variablen,  $D  = \{dom(v_1), ..., dom(v_n)\}$ die Menge der Werte der Variablen und $C = C_{KB} \cup REQ$, wobei $C_{KB}$ die oben beschriebene Wissensbasis und $REQ$ die Menge der Kundenanforderungen ist \citep{felferning14}.

\subsubsection{Konfigurationslösung}
Auf Grundlage des \ac{CSP} kann eine Konfigurationslösung formell definiert werden. Es handelt sich dabei um eine Instanziierung $I = \{v_1 = i_1, ..., v_n = i_n\}$, wobei $i_j$ ein Element aus $dom(v_1)$ ist. I ist vollständig (jede Variable bestitzt einen zugewiesenen Wert) und konsistent (erfüllt alle Constraints) \citep{falkner11}. Eine solche Lösung wird als korrekt bezeichnet \citep{soininen98}.

Eine korrekte Lösung kann durch ein UML-Instanz-Diagramm visualisiert werden.  Abbildung \ref{fig:notebookInstanceUML} zeigt eine korrekte Lösung der Notebook-Konfiguration. Dargestellt wird also eine mögliche Variantenausprägung. Anstatt der Komponententypen sind nur noch konkrete Instanzen enthalten.

\vspace{1em}
\begin{minipage}{\linewidth}
	\centering
	\includegraphics[width=1\linewidth]{Abbildungen/notebookInstanceUML.pdf}
	\captionof{figure}[notebookInstanceUML]{Visualisierung einer Konfigurationslösung als UML-Instanz-Diagramm}
	\label{fig:notebookInstanceUML}
\end{minipage}
\vspace{1em}

Lieferte eine Konfigurationsaufgabe mehr als eine korrekte Lösung, ist das Ergebnis eine Variantenmenge. Eine nicht erfüllbare Konfigurationsaufgabe führt hingegen zu einer leeren Lösungsmenge.

Diese Beschreibung einer Konfigurationslösung suggeriert, dass zu Beginn des Konfigurationsprozesses einmalig die Kundenanforderungen aufgenommen und daraufhin die Lösungsmenge ermittelt wird. Diese Form wird als statische Konfiguration bezeichnet. Demgegenüber erlaubt die interaktive Konfiguration das schrittweise Treffen und Revidieren von Entscheidungen \citep{hadzic04}. Bei der Verarbeitung einer Konfigurationsaufgabe dürfen die bisher getroffenen Entscheidungen nicht vergessen werden. Die Menge der Anwenderentscheidungen wird als \textbf{Konfigurationszustand} bezeichnet \citep{tactonTCsiteDevelopmentManual}. Der Konfigurator muss einen Mechanismus besitzen, bei der die Konfigurationsaufgabe in Bezug mit dem Konfigurationszustand gebracht wird.

\subsubsection{Konfigurationssysteme}
Der Konfigurator ist das System, welche die Schnittstelle zum Benutzer darstellt und den (interaktiven) Konfigurationsprozess durchführt. Es bekommt die Konfigurationsaufgabe als Eingabe und liefert als Ausgabe die Konfigurationslösung \citep{felferning14}. Konfiguratoren "[...] führen den Abnehmer durch alle Abstimmungsprozesse, die zur Definition des individuellen Produktes nötig sind und prüfen sogleich die Konsistenz sowie Fertigungsfähigkeit der gewünschten Variante" \citep{piller06}.

Nach \citet{piller06} besitzt ein Konfigurator drei Komponenten:
\begin{compactitem}
\item Die \textbf{Konfigurationskomponente} führt den Konfigurationsprozess durch. Sie wird auch als Konfigurationsengine bezeichnet \citep{tactonProductOverview}.
\item Die \textbf{Präsentationskomponente} erstellt eine Konfigurationsdarstellung in zielgruppenspezifischer Form. Daraus lässt sich ableiten, dass sie gleichzeitig als Schnittstelle zur Aufnahme der Kundenanforderungen dient.
\item Die \textbf{Auswertungskomponente} präsentiert der Konfiguration in einer Form, welche eine Interpretation der Variantenausprägung außerhalb des Konfigurators erlaubt. Dies können zum Beispiel Stücklisten, Konstruktionszeichnungen und Arbeitspläne sein.
\end{compactitem}

\textbf{Konfiguratorarten}\\
Konfiguratoren für die Erhebung komplexer Anforderungen technischer Systeme  müssen von Konfiguratoren für den Einsatz bei \ac{MC} unterschieden werden \citep{felferning14}. Erstere sind für den Experteneinsatz gedacht oder dienen nach \citet{piller06} als Vertriebskonfiguratoren der Unterstützung des Verkaufsgespräches.   Letztere werden von Kunden in einer Company-to–Customer Beziehung genutzt und werden auch als Mass Customization Toolkit bezeichnet. Diese sogenannte Selbstkonfiguration ist eine Vorraussetzung für \ac{MC}, indem der zeitkonsumierende Prozess der Erhebung der Kundenbedürfnisse auf die Seite des Kunden verlagert wird \citep{piller06}

Konfiguratoren können bei allen in Abschnit \ref{subssubsection:Produktklassifizierung} genannten Produktionskonzepten zum Einsatz kommen. Je nach Produktionskonzept erfüllen sie für den Anwender eine unterschiedliche Funktion. Bei \ac{PTO} erfüllt der Konfigurator eine Katalogfunktion, indem er den Anwender bei der Auswahl eines fertigen Produktes aus einer Produktpalette unterstützt. Bei \ac{ATO} verhält sich der Konfigurator wie ein Variantengenerator, der den Anwender bei der Auswahl der richtigen Variantenausprägung unterstützt. Wohlgemerkt: der Hersteller hat alle möglichen Variante vordefiniert, sie sind also herstellerspezifisch \citep{schomburg80}. Der Anwendungsfall \ac{MTO} ist ähnlich, jedoch werden Komponenten kundenspezifisch hergestellt oder regelbasiert konstruiert. Es wird von kundenspezifischen Varianten gesprochen \citep{schomburg80}. Bei \ac{ETO} besteht ein erheblicher Neukonstruktionsbedarf. Dies widerspricht der Definition der Konfiguration als Designaktivität aus Abschnitt \ref{subsubsection:begriffsuberblick}
 - die Spezifikation der beteiligten Objekte ist nicht vollständig bekannt. Konfiguration können hier nur einen begrenzt Beitrag leisten. Aus dieser Erläuterung lässt sich Ableiten, dass das Haupteinsatzgebiet von Konfiguratoren im \ac{ATO}/\ac{MTO} Umfeld liegt.

\textbf{Zwischenfazit}\\
In Abschnitt \ref{subssubsection:Produktklassifizierung} wurde dargestellt, wie bestimmte Produktionskonzepte die Herstellung individualisierter Produktvarianten bei gleichzeitiger Lagerfertigung ermöglichen. Produkte werden mit dem Ziel gestaltet, so individuell und auftragsunabhängig wie möglich zu sein. Damit wurde eine der Schlüsselfaktoren für die Ermöglichung der hybriden Wettbewerbsstrategie \ac{MC} erläutert. Diese verbindet die Vorteile effizienter Massenproduktion mit denen
der kundenspezifischen Einzelfertigung\citep{piller98}. \ac{MC} resultiert in Variantenvielfalt und damit in Produktkomplexität. In Abschnitt \ref{subssubsection:Produktkonfiguration} wurde die Funktionsweise von Konfiguratoren vorgestellt, mit welcher sie zur Beherrschung der Produktkomplexität beitragen.


\pagebreak
\subsection{Webservices}

Das \citet	{w3c04} definiert Webservices lose als:

\begin{quote}
\glqq [...] a software system designed to support interoperable machine-to-machine interaction over a network\grqq
\end{quote}

Die Definition schließt die Kommunikation heterogener Systeme ein. \glqq Zwischen Systemen\grqq{} differenziert gleichzeitig klar zur klassischen Verwendung eines Programms, bei der ein (menschlicher) Nutzer mit einem System kommuniziert. \citet{tilkov11} bemerkt, dass Web Service damit sehr weich definiert ist; \glqq nämlich eigentlich gar nicht\grqq{}. Fest steht, dass hier ein Service einen Dienst anbietet, der von einem Clienten über Webtechnologien angesprochen werden kann. Webservices sind demzufolge eine Möglichkeit zur Realisierung von Integrationsszenarien webbasierter Systeme.

\vspace{1em}
\begin{minipage}{\linewidth}
	\centering
	\includegraphics[width=0.7\linewidth]{Abbildungen/clientServerKommunikation.png}
	\captionof{figure}[clientServerKommunikation.png]{Generische Client-Server Kommunikation bei Webservices}
	\label{fig:clientServerKommunikation.png}
\end{minipage}
\vspace{1em}

Abbildung \ref{fig:clientServerKommunikation.png} entspricht im wesentlichen der klassischen Client-Server Kommunikation im Web. Exemplarisch werden XML-Daten übertragen, was die zugrunde liegende Idee der Webservices illustriert: die Übertragung anderer Daten als Webseiten mittels HTTP.

\citet{wilde11} reden von zwei etablierten \glqq Geschmäkern\grqq{} (flavors) in der Webservice-Welt: SOAP und REST. Die erste Geschmacksrichtung bedeutet Webservices \glqq auf Basis von SOAP, WSDL und den WS-*Standards - bzw. [...] deren Architektur\grqq{} \citep{tilkov11}. Hier wird also ein XML-basierter Technologiestack beschrieben. REST hingegen ist ein Archikturstil, der 2000 in der Dissertation von \citeauthor{fielding00} vorgestellt wurde. Der Versuch, beide Varianten direkt gegenüberstellen stellen zu wollen, ist ein \glqq [...] klassischer Apfel-Birnenvergleich: ein konkretes XML-Format gegen einen abstrakten Architekturstil\grqq{} \citep{tilkov11}.

Vor einer detaillierteren Diskussion von SOAP und REST wird zur Einordnung eine grundlegende Unterscheidung der Ansätze vorgestellt. Gemeinsam ist beiden, dass HTTP als Transportprotkoll zur Übertragung der Frage (Request) verwendet wird, die vom Server (Response) beantwortert werden soll. HTTP wiederum besteht aus einem Header und einem Entity-Body zur Übertragung von Daten. \citet{richardson07} haben zwei Leitfragen herausgearbeitet, die von den jeweiligen Ansätzen unterschiedlich beantwortet werden: wo in diesem Paket sagt der Client dem Service, mit welchen Daten (Fokusinformation) was (Methodeninformation) gemacht werden soll?

Die Fokusinformation sagt aus, für welche Datenelemente sich der Client interessiert (z.B. ein Artikel eines Onlineshops). Bei REST ist dies der URI zu entnehmen (z.B. http://onlineshop.com/artikel/pc). Bei SOAP steht diese Information in einer XML-Datei welche Entity-Body übertragen wird; die sogenannte Payload. Die Methodeninformation sagt aus, was mit dem identifizierten Datenelement geschehen soll (Bsp: lege einen neuen PC-Artikel an). Bei REST steht dies im Methodenfeld der HTTP-Headers, bei SOAP wieder im Entity-Body. Daraus lässt sich als grundlegender Unterschied ableiten: SOAP verwendet HTTP nur als Transportprotokoll, REST auch dessen Ausdruckskraft \citep{wilde11}.

\subsubsection{SOAP}
Bei SOAP-Web Services wird ein \ac{RPC} durchgeführt. Dabei handelt es sich um eine generelle Technik zur Realisierung von Systemverteilung. Ein System ruft die Funktion eines Systems aus einem anderen Adressraum auf. SOAP ist ein XML-basiertes Umschlagsformat, welches wiederum die Beschreibung eines Methodenaufrufs in XML-Form enthält. Bei SOAP-Web Services werden also \ac{RPC}s über HTTP getunnelt \citep{wilde11}. Das ist Konvention, aber keine Notwendigkeit: der SOAP-Umschlag ist Transportunabhängig, könnte also auch von anderen Protokollen als HTTP übertragen werden \citep{tilkov11}. Solange es sich bei dem Transportprotokoll um eine Webtechnologie handelt, wird die Webservicedefinition nicht verletzt.

Wie die Beschreibung des \ac{RPC} aussehen muss, definiert die \ac{WSDL}. Jeder SOAP basierte Service stellt eine machinenverarbeitbare \ac{WSDL}-Datei bereit. Darin werden die aufrufbaren Methoden, deren Argumente und Rückgabetypen beschrieben. Außerdem werden Schemata der XML-Dokumente festgehalten, die der Service akzeptiert und versendet \citep{richardson07}.

Es existieren eine Vielzahl von Middleware-Interoperabilitätsstandards, die mit dem \glqq WS-\grqq{} Prefix versehen sind. Diese sind \glqq XML-Aufkleber\grqq{} für den SOAP-Umschlag, die HTTP-Headern entsprechen \citep{richardson07}. Sie erweitern die Ausdrucksmöglichkeit des SOAP-Formats \citep{wilde11}. Beispielsweise erlaubt WS-Security die Berücksichtigung von Sicherheitsaspekten bei der Client-Server Kommunikation. Eine Übersicht der existierenden Standards ist dem Wiki für Webservices \citet{webServiceWiki09} zu entnehmen.

\subsubsection{REST}
\begin{quote}
\glqq Eine Architektur zu definieren bedeutet zu entscheiden, welche Eigenschaften das System haben soll, und eine Reihe von Einschränkungen vorzugeben, mit denen diese Eigenschaften erreicht werden können.\grqq{} \citep{tilkov11}
\end{quote}

Dies ist in der Dissertation von \citeauthor{fielding00} geschehen, in der REST als Architekturstil definiert wird. Ein Architekturstil ist ein stärkerer Abstraktionsgrad als eine Architektur. Beispielsweise ist das Web eine HTTP-Implementierung von REST \citep{tilkov11}. Tatsächlich wurden die Einschränkungen von REST aber dem Web entnommen, indem \citeauthor{fielding00} es post-hoc als lose gekoppeltes, dezentralisiertes Hypermediasystem konzeptualisiert \citep{wilde11} und dann von diesem Konzept abstrahiert hat. Einen Webservice nach dem REST-Architekturstil zu implementieren passt ihn dem Wesen des Webs an und nutzt dessen stärken\citep{tilkov11}.

Entsprechend \citeauthor{tilkov11}s Architekturdefinition werden im Folgenden die Einschränkungen von REST sowie die daraus resultierenden Eigenschaften besprochen.

\textbf{Einschränkungen}\\
Einschränkungen sind - in eigenen Worten - Implementierungskriterien. Während \citeauthor{fielding00} in seiner theoretischen Abhandlung explizit vier solcher Kriterien nennt, basiert die folgende Auflistung auf der praxiserprobten Variante der Sekundärliteratur \citep{wilde11, tilkov11}.

\begin{compactitem}
\item \textbf{Ressourcen mit eindeutiger Identifikation}: \glqq Eine Ressource ist alles, was wichtig genug ist, um als eigenständiges Etwas referenziert zu werden\grqq{} \citep{richardson07}. Identifiziert werden sie im Web durch URIs, die einen globalen Namensraum darstellen.  Es ist hervorzuheben, dass Ressourcen nicht das gleiche sind wie die Datenelemente aus der Persistenzschicht einer Anwendung. Sie befinden sich auf einem anderen Abstraktionsniveau. Beispiel: eine Warenkorbressource kann eine Auflistung von Artikeln sein, welche allerdings nicht einzeln als Ressource ansprechbar sind. \citeauthor{tilkov11} nimmt in diesem Zusammenhang eine Typisierung von Ressourcen vor. Von den sieben verschiedene Ressourcentypen sind folgende im Rahmen der Fragestellung interessant:
\begin{enumerate}[a.]
\item Bei einer \textbf{Projektion} wird die Informationsmenge verringert, indem eine sinnvolle Untermenge der Attribute einer abgerufenen Ressource gebildet wird. Zweck ist die Reduktion der Datenmenge. Beispiel: Weglassen der Beschreibungstexte von Warenkorbartikeln.
\item Die \textbf{Aggregation} ist das Gegenteil. Hier werden Attribute unterschiedlicher Ressourcen zur Reduktion der Anzahl notwendiger Client/Server Interaktionen zusammengefasst. Beispiel: Hinzufügen der Versandkosten beim Abruf der Warenkorbartikel.
\item \textbf{Aktivitäten} sind Ressourcen, die sich aus Prozessen ergeben, wie etwa ein Schritt innerhalb einer Verarbeitung. Beispiel: Ein Schritt einer nicht abgeschlossenen Konfiguration.
\end{enumerate}
\item \textbf{Hypermedia} beschreibt das Prinzip verknüpfter Ressourcen. So wird dem Client ermöglicht, neue Ressourcen zu entdecken oder bestimmte Prozesse anzustoßen. Beispiel: Zur einer Bestellbestätigungsressource wird der zugehörige Stornierungslink hinzugefügt.
\item \textbf{Standardmethoden/uniforme Schnittstelle}: Oben wurde beschreiben, dass jede Ressource durch (mindestens) eine ID identifiziert wird. Jede URI unterstützt dabei den gleichen Methodensatz, welche mit den HTTP-Methoden korrespondieren. Das bedeutet - übertragen auf die objektorientierte Programmierung: jedes Objekt implementiert das gleiche Interface. Folgende Teilmenge der neun verfügbaren HTTP-Methoden finden in der Literatur am häufigsten Erwähnung:
\begin{enumerate}[a.]
\item \textbf{GET:} Das Abholen einer Ressource.
\item \textbf{PUT:} Das Anlegen oder Aktualisieren einer Ressource. Je nachdem, ob unter dieser URI bereits eine Ressource existiert.
\item \textbf{POST:} Bedeutet im engeren Sinne das Anlegen einer Ressource unter einer URI, die vom Service bestimmt wird. Im weiteren Sinne kann durch Post ein Prozess angestoßen werden.
\item \textbf{Delete:} Das Löschen einer Ressource.
\end{enumerate}

\vspace{1em}
\begin{minipage}{\linewidth}
	\centering
	\includegraphics[width=0.7\linewidth]{Abbildungen/restMethoden.jpg}
	\captionof{figure}[restMethoden]{HTTP-Methoden und ihre Eigenschaften
	\footnote{Relevante Attribute im Rahmen der Fragestellung: \glqq sicher\grqq{} bedeutet nebenwirkungsfrei, d.h. kein Ressourcenzustand ändert sich durch diese Methode. \glqq Idempotent\grqq{} bedeutet, dass das Resultat der Methode bei Mehrfachausführung das gleiche ist. \glqq Identifizierbare Ressource\grqq{} bedeutet, dass die URL garantiert eine Ressource identifiziert.}
	(Quelle: \citet{tilkov11})}
	\label{fig:restMethoden}
\end{minipage}
\vspace{1em}

Abbildung \ref{fig:restMethoden} fast die Eigenschaften der Methoden aus der HTTP-Spezifikation 1.1 zusammen. Die Implementierung einer Methode muss dem erwarteten Verhalten aus dieser Spezifikation entsprechen. Die Praxis zeigt, dass nur die Methoden unterstützt werden, die für die jeweilige Ressource sinnvoll sind. Abbildung \ref{fig:restMethoden} macht außerdem klar, dass es für POST keinerlei Garantien gibt. Da nicht eindeutig ist, ob über POST eine Ressource erstellt oder ein Prozess angestoßen wird, sehen \citeauthor{richardson07} hierin eine Verletzung der uniformen Schnittstelle. Dies bedeutet in der Praxis: was bei einem Post passiert, ist nicht der HTTP-Spezifikation, sondern der API-Beschreibung des Webservice zu entnehmen.
\item \textbf{Ressourcen und Repräsentationen:} Beschreibt die Darstellungen einer Ressource in einem definierten Format. Der Client bekommt nie die Ressource selbst, sondern nur eine Repräsentation derer zu sehen. In der Praxis wird meist eine  serialisierte Variante eines Objektes als JSON zur Verfügung gestellt. Beispiel: Bereitstellung einer Bestellbestätigung als PDF und HTML.
\item \textbf{Statuslose Kommunikation} bedeutet die Nichtexistenz eines serverseitig abgelegten, transienten, clientspezifischen Status über die Dauer eines Requests hinweg. Der Service benötigt also nie Kontextinformationen zur Bearbeitung eines Requests. Beispiel: Ein Warenkorb wird nicht in einem Sessionobjekt, sondern als persistentes Datenelement gehalten.
\end{compactitem}

Diese Auflistung legt folgende Frage nahe: ist ein Webservice nur dann REST-konform, wenn alle Kriterien erfüllt werden? Was ist mit einem Webservice, der allen Einschränkungen gerecht wird, jedoch Ressourcen nur als JSON ausliefert - ein in der Praxis häufig anzutreffender Fall. Und dennoch ein Verstoß gegen die Forderung nach unterschiedlichen Repräsentationen. Aus diesem Grund existiert das \glqq Richardson Maturity Model\grqq{}, welches die abgestufte Bewertung eines Webservices nach dessen REST-Konformität erlaubt. Es wird im Auswertungsteil vorgestellt und zur Evaluierung der Implementierung genutzt.

\textbf{Eigenschaften}\\
Aus den vorgestellten Kriterien resultieren folgende Eigenschaften \citep{tilkov11},  welche die Vorteile REST-basierter Webservices gegenüber der SOAP-Konkurrenz darstellen \citep{richardson07}:

\begin{compactitem}
\item \textbf{Lose Kopplung:} Beschreibt isolierte Systeme mit größtmöglicher Unabhängigkeit, die über Schnittstellen miteinander kommunizieren. Hierzu tragen die Standardmethoden bei.
\item \textbf{Interoperabilität:} Beschreibt die Möglichkeit der Kommunikation von Systemen unabhängig von deren technischen Implementierung. Dies ergibt sich durch die Festlegung auf Standards. Bei der Anwendung von REST auf Webservices sind dies die Webstandards (z.B. HTTP, URIs).
\item \textbf{Wiederverwendbarkeit}: Jeder Client, der die Schnittstelle eines REST-basierten Service verwenden kann, kann auch jeden anderen beliebigen REST-basierten Service nutzen - vorausgesetzt, das Datenformat wird von beiden Seiten verstanden.
\item \textbf{Performance und Skalierbarkeit}: Im Ideal sollen Webservices schnell antworten, unabhängig von der Anzahl von Anfragen in einem definierten Zeitraum. Dies wird durch Cachebarkeit (siehe HTTP-Methodenspezifikation) und Zustandslosigkeit erreicht. Da der Service keinen clientspezifischen Kontext aufbauen muss, müssen aufeinanderfolgende Requests nicht vom gleichen System beantwortet werden.
\end{compactitem}
\pagebreak

\subsection{eCommerce}

Im Folgenden wird durch die Charakterisierung des Begriffs eCommerce ein Andwendungsrahmen für eShops hergestellt. Deren softwaretechnische Umsetzung wird durch eShop-Systeme realisiert. Durch eine Kategorisierung der Systeme nach Anbieterstrategie wird abschließend die Menge der Open-Source-Lösungen für eine Konfiguratorintegration identifiziert.

\subsubsection{Anwendungsrahmen}
eShops gehören zur Domäne des elektronischen Handels (eCommerce). eCommerce ist \glqq die elektronisch unterstützte Abwicklung von Handelsgeschäften auf der Basis der Internet\grqq{} \citep{schwarze02}. Je nachdem, welche Marktpartner an dem Handelsgeschäft teilnehmen, werden verschiedene Formen des eCommerce unterschieden. Die in Abbildung \ref{fig:eCommerceGrundformen} fett hervorgehobenen Varianten werden von \citet{meier12} als \glqq die zwei Geschäftsoptionen des eCommerce\grqq{} bezeichnet: \ac{B2C} und \ac{B2B}. Bei \ac{B2C} erfolgt der Handel von Produkten und Dienstleistungen zwischen Unternehmen and Endverbraucher, bei \ac{B2B} zwischen Unternehmen.

\vspace{1em}
\begin{minipage}{\linewidth}
	\centering
	\includegraphics[width=0.7\linewidth]{Abbildungen/eCommerceGrundformen.jpg}
	\captionof{figure}[eCommerceGrundformen]{Grundformen des eCommerce nach Marktpartnern (Quelle: \citet{schwarze02})}
	\label{fig:eCommerceGrundformen}
\end{minipage}
\vspace{1em}

Für die Umsetzung von eCommerce existieren unterschiedliche Geschäftsmodelle. \citet{timmers98} nennt 11 verschiedene Formen. eShops sind eine eine davon. Es handelt sich dabei um ein \glqq Geschäftsmodell der Angebotsveröffentlichung, bei dem ein Anbieter seine Waren oder Dienstleistungen über das Web den Nachfragern offeriert\grqq{} \citep{bartelt00}.

Ein eShop bildet den traditionellen Einkaufsvorgang nach: Kunden können mittels einer Katalog- oder Suchfunktion über den Produktbestand navigieren. Produkte können ausgewählt und ausführliche, mit Medien angereicherte Beschreibungen abgerufen werden. Wunschartikel werden einem virtuellen Warenkorb hinzugefügt. Ist die Produktauswahl abgeschlossen, begibt sich der Kunde zur \glqq Kasse\grqq{}, wo die Zahlungsmodalitäten erledigt werden \citep{boles00}. Ein eShop beschreibt das Geschäftsmodell, jedoch noch nicht dessen Umsetzung als Softwaresystem. Diese wird als eShop-System bezeichnet \citep{boles00} und im Folgenden behandelt.

\subsubsection{eShop-Systeme}
\glqq eShop-Systeme sind Software-Systeme, die den Aufbau, die Verwaltung und den Einsatz von eShops unterstützen\grqq{} \citep{boles00}. Abbildung \ref{fig:eShopGrobarchitektur} zeigt die Grobarchitektur eines eShop-Systems nach \citeauthor{meier12}. Darin wird die Unterteilung zwischen Storefront und Backfront deutlich, welche in der Terminologie realer Shopsysteme als Front- und Backend bezeichnet werden \citep[vgl.][]{shopwareDoku}. Das Frontend ist der Interaktionsraum der Kunden, das Backend der administrative Bereich des Shopbetreibers.

\vspace{1em}
\begin{minipage}{\linewidth}
	\centering
	\includegraphics[width=0.7\linewidth]{Abbildungen/eShopGrobarchitektur.jpg}
	\captionof{figure}[eShopGrobarchitektur]{Grobarchitektur eines eShop-Systems (Quelle: \citet{meier12})}
	\label{fig:eShopGrobarchitektur}
\end{minipage}
\vspace{1em}

Die Hauptaufgaben eines eShop-Systems sehen \citet{boles00} in den Bereichen Merchandising (z.B. Management von hierarchisch strukturierten Produktkatalogen, Beeinflussung des Shopdesigns), Auftragsbearbeitung (z.B. Festlegung der Abarbeitungs-Pipeline, Integration von Bezahlverfahren) und Sonstiges (z.B. die Kopplung mit externen ERP-Systemen). Der konkrete Funktionsumfang hängt vom  gewählten eShop-System ab.

Die Systeme sind nach Strategie der Anbieter kategorisierbar:
\begin{compactitem}
\item \textbf{Open-Source}-Systeme sind kostenlos verfügbar. Sie bieten völlige Gestaltungsfreiheit, aber keinen Herstellersupport. Die Dokumentationen sind schwächer und der Funktionsumfang geringer als bei kostenpflichtigen Alternativen. Andererseits existieren Communities, die Unterstützung bieten und die Entwicklung von Erweiterungen vorantreiben \citep{stahl15}. Beim kommerziellen Handel der modularen Erweiterungen auf shopspezifischen Stores liegt auch eine der wesentlichen Erlösquellen der Open-Source Strategie (z.B. der Addon Marketplace der \citeauthor{prestashopAddons} oder das Extensionverzeichnis der \citeauthor{opencartExtensions}).
\item \textbf{Kauf-Lösungen} können kostenpflichtig lizensiert werden (z.B. \citeauthor{shopwarePricing}). Sie bieten Herstellersupport, zusätzliche Dienstleistungen (z.B. Installation des Shops) und einen höheren Funktionsumfang (z. B. Schnittstellen zu verschiedenen Warenwirtschaftssystemen oder Zahlungsdienstleistern) \citep{stahl15}. Die Hersteller bieten verschiedene Editionen mit teilweise erheblichen Preisunterschieden an (Bsp.: die Preisdifferenz der Magento Enterprise Edition zu Enterprise Premium liegt bei über 35.000 \$, \citealp[vgl][]{fwpShop}).
\begin{enumerate}[a.]
\item Im Rahmen eines Dual-License-Modells ist eine Open-Source \textbf{Community Edition} Teil des Editionsspektrums \citep{t3n14} (vgl. das Shopangebot der \citeauthor{magentoShops}, \citeauthor{shopwarePricing} oder \citeauthor{oxidShops}). Durch den offenen Quellcode existiert auch hier der Handel modularer Erweiterungen, von dem auch die kostenpflichtigen Varianten profitieren (vgl. der Plugin Store der \citeauthor{shopwarePluginStore}). Die Codebasis aller Editionen ist gleich. Daher kann später zu einer Kauf-Lösung migriert werden. Das bietet Flexibilität für wachsende Shopanforderungen.
\end{enumerate}
\item \textbf{Miet-Shops} entsprechen einer Cloud-Lösung als Software-as-a-Service (z.B. \citeauthor{stratoWebshops}, \citeauthor{shopify15}). Die technische Infrastruktur wird vom Provider zur Verfügung gestellt. Systemwartung, Bereitstellung der Shopsoftware und Hosting werden unter dem Mietpreis abgerechnet. \citet{stahl15} bewertet diese Variante als Einstiegslösung mit geringer Gestaltungsfreiheit.
\item \textbf{Eigenentwicklungen} eignen sich für individuelle Bedürfnisse, wenn die Standardsysteme die Anforderungen nicht mehr erfüllen \citep{stahl15, graf14}.
\end{compactitem}

Aus der Liste sind die (zumindest initial) kostenfreien Varianten ersichtlich: reine Open-Source eShop-Systeme sowie die Community-Editionen der Dual-License Modelle. Eine Anbieterübersicht ist \citet{t3n14} zu entnehmen. Eine Kategorisierung der Systeme nach Anforderungsklassen ist \citet{graf14} zu entnehmen.

\pagebreak
\section{Analyse}

Die Tacton Systems AB (Tacton) wurde 1998 als Spin-Off des Schwedischen Instituts für Informatik (SICS) gegründet \citep{tactonProductOverview}. In der Forschungseinrichtung wurde als Resultat der Untersuchungen im Bereich Wissensbasierte Systeme und Künstliche Intelligenz der Tacton Produktkonfigurator entwickelt. Dieser interaktive Konfigurator ist die Basis der verschiedenen Produkte der Firma.  \citep{tactonAbout}

Tacton bietet Lösungen im Bereich Vertriebskonfiguration und Design Automation (Automatisierung der Konstruktion in CAD-Systemen). Weitere einführende Worte...

\subsection{Konfigurationsmodell}
\label{subsection:Konfigurationsmodell}
Das in Abschnitt \ref{subsubsection:wissenrepraesentation} vorgestellte Visualisierungskonzept abstrahiert Konfigurationswissen in einen Struktur- und einen Regelteil. Im Tacton-Konfigurationsmodell wird ebenfalls abstrahiert, jedoch in andere Domänen \citep{tactonModeling}:

\begin{enumerate}[(a)]
\item \label{strukturinformation} Strukturinformation: wie ist das Produkt hierarchisch aufgebaut?
\item \label{komponenteninformation} Komponenteninformation: aus was ist es aufgebaut?
\item \label{constraintinformationen} Constraintinformationen: wann ist das Produkt korrekt?
\item \label{ausfuehrungsinformationen} Ausführungsinformationen: welche Fragen bekommt der Nutzer in welcher Reihenfolge gestellt?
\end{enumerate}

Dabei sind zwei Sachverhalte feststellbar:
\begin{enumerate}[(1)]
\item \label{enum:componentsConfiguration} Typisch für einen modellbasierten Konfigurator wird Produktwissen und Problemlösungswissen bei der Wissensmodellierung separiert. Beispiel: ein Constraint, der die Kompatibilität bestimmter Betriebssystems mit einer bestimmten Anzahl Prozessorkerne ausdrückt, soll wirken, auch ohne die konkreten CPU-Typen (z.B. Intel i7) zu kennen. Darum ist die Rede von generischen Constraints: sie beziehen sich auf alle Komponenten eines Typs \citep{felferning14}.
\item \label{enum:execution} Im Konfigurationsmodell wird auch die Nutzerinteraktion festgelegt. Damit geht der Funktionsumfang eines Tacton-Modells über die Definition aus Abschnitt \ref{subsubsection:begriffsuberblick} hinaus.
\end{enumerate}

Bei der UML-Wissensrepräsentation aus Abschnitt \ref{subsubsection:wissenrepraesentation} wurden die Domänen \eqref{komponenteninformation} und \eqref{strukturinformation} als Diagramm ausgedrückt sowie \eqref{constraintinformationen} als Tabelle. Tacton wählt eine andere Aufteilung. \eqref{komponenteninformation} wird unter dem Begriff \glqq Components\grqq{} umgesetzt, \eqref{strukturinformation} und \eqref{constraintinformationen} unter dem Begriff \glqq Configuration\grqq{}. Components und Configuration setzen gemeinsam den Sachverhalt \eqref{enum:componentsConfiguration} um und werden im Folgenden besprochen. Daraufhin wird die \glqq Execution\grqq{} besprochen, welche Sachverhalt \eqref{enum:execution} umsetzt.


\subsubsection{Components und Configuration}

\vspace{1em}
\begin{minipage}{\linewidth}
	\centering
	\includegraphics[width=1\linewidth]{Abbildungen/tactonModellHighLevel.pdf}
	\captionof{figure}[tactonModellHighLevel]{High-Level Architektur des Tacton-Konfigurationsmodells}
	\label{fig:tactonModellHighLevel}
\end{minipage}
\vspace{1em}

Abbildung \ref{fig:tactonModellHighLevel} zeigt das High-Level-Konzept der Modellarchitektur. Unter 'configuration' wird die Produktstruktur als hierarchischer Baum  von 'Part'-Objekten dargestellt. Jeder Part kann Constraints enthalten, die sich auf den Knoten selbst und alle seine Kinder beziehen. Ein Part ist ansonsten nur ein Komponentenplatzhalter. Noch ist keine Information darüber hinterlegt, welches Bauteil dort eigentlich verkörpert wird. Abbildung \ref{fig:tactonModellHighLevelNotebook} veranschaulicht das Konzept am Notebook-Beispiel.

\vspace{1em}
\begin{minipage}{\linewidth}
	\centering
	\includegraphics[width=0.7\linewidth]{Abbildungen/tactonModellHighLevelNotebook.pdf}
	\captionof{figure}[tactonModellHighLevelNotebook]{Part-Struktur der Notebook-Konfiguration}
	\label{fig:tactonModellHighLevelNotebook}
\end{minipage}
\vspace{1em}

Die Informationen über die eben erwähnten Bauteile (d.h. Komponententypen) werden isoliert unter 'components' definiert (Abbildung \ref{fig:tactonModellHighLevel}). ????Analog zu den Komponententypen und Komponenten aus Abschnitt \ref{subsubsection:wissenrepraesentation} wird ein Bauteil in 'Component Classes' und 'Components' abstrahiert. Hier wurde einfach eine andere Terminologie gewählt????. Eine Component Class (z.B. ein 'RAM') wird durch Eigenschaften beschrieben, die als Features bezeichnet werden. Jedes Feature besitzt einen Namen (z.B. 'memory') und einen Wertebereich (z.B. [2GB, 4GB]), welche als Domain bezeichnet wird. Wertebereiche können Integer, Float, Boolean, andere Component Classes oder selbstdefinierte Enumerationen sein. Components (z.B. eine 'SSD') sind das, was eine Component Class (z.B. ein 'Storage') konkret sein kann. Sie übernehmen alle Features der übergeordneten Component Class und besitzen konkrete Werte ('Values') aus dem jeweiligen Wertebereich.

\vspace{1em}
\begin{minipage}{\linewidth}
	\centering
	\includegraphics[width=1\linewidth]{Abbildungen/tactonModellLowLevel.pdf}
	\captionof{figure}[tactonModellLowLevel]{Zuordnung von Parts und Component Classes.}
	\label{fig:tactonModellLowLevel}
\end{minipage}
\vspace{1em}

Abbildung \ref{fig:tactonModellLowLevel} zeigt die Zuordnung zwischen Parts und Component Classes via Verzeigerung. Einem Part wird eine Component Class durch eine 'realized-by' Beziehung zugeordnet. Dem Part werden dabei die Features der entsprechenden Component Class vererbt (türkis dargestellt). Nur werden sie zur besseren Differenzierung beim Part nicht mehr als Features, sondern als Attribute bezeichnet. Für einen Part können auch noch zusätzliche Attribute definiert werden ('additional Attributes'), falls notwendig. Die Angabe 'Instances' (grün dargestellt) entspricht den Kardinalitäten in Abschnitt \ref{fig:tactonModellLowLevel}. Constraints (gelb dargestellt) werden als logische Ausdrücke formuliert. Sie bestehen aus Attributwerten, die mit mathematischen Zeichen in Relation gebracht werden.

\vspace{1em}
\begin{minipage}{\linewidth}
	\centering
	\includegraphics[width=1\linewidth]{Abbildungen/tactonModellLowLevelNotebook.pdf}
	\captionof{figure}[tactonModellLowLevelNotebook]{Part-Struktur der Notebook-Konfiguration}
	\label{fig:tactonModellLowLevelNotebook}
\end{minipage}
\vspace{1em}

Abbildung \ref{fig:tactonModellLowLevelNotebook} veranschaulicht die 'realized-by' Beziehung anhand eines Ausschnitt aus der Notebook-Konfiguration.  Der Part übernimmt die Attribute 'memory' und 'price'. Das zusätzliche Attribut 'type' kapselt beispielsweise die Information über die jeweiligen Components (z.B. Small RAM). Der Constraint veranschaulicht exemplarisch, wie die Inkompatibilität mit dem Betriebssystem vom Typ 'OSX Yosemite' formuliert werden würde.

\subsubsection{Execution}
\label{subsubsection:Execution}
Es wurde dargestellt, wie das Konfigurationswissen im Tacton Konfigurationsmodell definiert wird. Dadurch ist aber noch nicht gesagt, welche Entscheidungen ein Nutzer während des Konfigurationsprozesses treffen kann. Nicht jeder Part und nicht jedes Attribut muss eine relevante Wahl darstellen. Vielleicht sollen dem Nutzer sogar Fragen auf einem anderen Abstraktionsniveau als auf Komponentenebene gestellt werden. Statt \glqq Soll eine HD oder eine SSD als Festplatte in das Notebook eingebaut werden?\grqq{} kann auch gefragt werden: \glqq Möchten Sie viel Speicherplatz oder einen schnellen Speicherzugriff?\grqq.

Die Interaktionsspielraum des Anwenders mit der Konfiguration wird unter dem Begriff 'Execution' zusammen gefasst. Dabei wird nicht einfach nur eine Menge an Optionen festgelegt, die dem Nutzer am Ende als Liste präsentiert wird. Stattdessen werden die Optionen hierarchisch Strukturiert.

\vspace{1em}
\begin{minipage}{\linewidth}
	\centering
	\includegraphics[width=1\linewidth]{Abbildungen/tactonModellExecution.pdf}
	\captionof{figure}[tactonModellExecution]{Generische Executionstruktur}
	\label{fig:tactonModellExecution}
\end{minipage}
\begin{minipage}{\linewidth}
	\centering
	\includegraphics[width=0.8\linewidth]{Abbildungen/tactonModellExecutionNotebook.pdf}
	\captionof{figure}[tactonModellExecutionNotebook]{Exemplarische Executionstruktur einer Notebook-Konfiguration}
	\label{fig:tactonModellExecutionNotebook}
\end{minipage}
\vspace{1em}

Abbildung \ref{fig:tactonModellExecution} zeigt die generische Struktur der Execution als Baum. Abbildung \ref{fig:tactonModellExecutionNotebook} veranschaulicht das Konzept durch die exemplarische Umsetzung der Execution der Notebook-Konfiguration.
\begin{compactitem}
\item Die Wurzel wird als \textbf{Applikation} bezeichnet und kapselt den Konfigurationsprozess aus Anwenderperspektive.\\
Beispiel: Eine Notebook-Konfiguration.
\item Die nächste Knotenebene wird als \textbf{Steps} bezeichnet. Sie legen fest, in welcher Reihenfolge die Optionen präsentiert werden. Das ist in verschiedenen Fällen sinnvoll. Zum Beispiel dann, wenn komplexe Probleme in mehrere Schritte unterteilt oder Werte für spätere Schritte gesetzt werden sollen.\\
Beispiel: In Step 1 legt der Anwender eine Preisgrenze fest. In Schritt 2 wird die technische Spezifikation getroffen.
\item  Nun folgen $1..n$ Knotenebenen, die als \textbf{Groups} bezeichnet werden. Sie fassen Optionen zu logischen Einheiten zusammen. Die in Gruppen angeordneten Optionen müssen nicht in einer bestimmten Reihenfolge beantwortet werden.
\begin{enumerate}[(a)]
\item Die oberste Group-Ebene \textbf{Top-Level-Groups} ist obligatorisch.\\
Beispiel: Eine 'Speicher'-Group in Step 2.
\item Die Unterteilung in weitere Groups ist optional. Es kann beliebig tief geschachtelt werden.\\
Beispiel: unter 'Speicher' wird die Group 'Massenspeicher' angelegt, um sie vom 'RAM' zu unterscheiden.
\end{enumerate}
\item Die tatsächlichen Optionen werden als \textbf{Fields} bezeichnet. Aus Sicht des Anwenders besteht ein Field aus einer Beschreibung und einem Interaktionselement, über das eine Wahl getroffen werden kann. Eine Wahl aus Anwendungssicht bedeutet: der Wert eines Part-Attributs. Damit ist ein Feld mit einem Part-Attribut verknüpft, so wie Part-Attribute mit den Features der Component Classes verknüpft sind. In der Terminologie der Execution ist das Attribut der \textbf{Parameter}, sein Wert der \textbf{Value}. Ein intuitiver Begriff für Value ist Option.\\
Folgende Interaktionselemente stehen zur Verfügung:
\begin{enumerate}[(a)]
\item \textbf{Menu:} Wahl aus einem Wertebereich, z.B. über ein Dropdownmenü realisiert.\\
Beispiel Massenspeicher:\\
Parameter = Storage-Part.type,  Domain = ['HD', 'SSD']\\
Beispiel CPU-Kerne:\\
Parameter = CPU-Part.cores, Domain = [1..4].
\item \textbf{Number:} Eingabe eines eigenen Wertes in ein Textfeld.\\
Beispiel Preisgrenze (Eingabe in Schritt 1):\\
 Parameter = Notebook-Part.maxPrice, Domain = [integer]
\item \textbf{Label:} Anzeige eines Wertes aus Informationsgründen.\\
Beispiel Aktueller Preis (Anzeige in Schritt 2):\\
Parameter = Notebook-Part.price, Domain = [integer]
\end{enumerate}
\end{compactitem}

Das Modell wird in einer XML-basierten Format mit der Dateiendung '.tcx' abgelegt. Die Entwicklung einer solchen Datei wird von entsprechenden Programmen unterstützt. Hierfür kann zum Beispiel \glqq TCstudio\grqq{} genutzt werden, welches eine grafische Oberfläche zur Modellentwicklung bietet \citep{tactonAbout}.

\textbf{Zwischenfazit}\\
Das Tacton-Modellierungskonzept wurde vorgestellt. Neben dem Konfigurationswissen werden darin auch die Optionen des Anwenders definiert. Diese werden durch Steps in eine Reihenfolge gebracht und durch Groups logisch gegliedert. Die  Interaktionsstruktur muss jedoch noch dargestellt werden. TCsite realisiert eine entsprechende Darstellung. Die Anwendung, mit besonderem Fokus auf die Konfigurationsoberfläche, wird im Folgenden analysiert.

\subsection{TCSite}

\textbf{TCsite} ist ein webbasierter Vertriebskonfigurator. Er könnte theoretisch von Endkunden genutzt werden, bildet aber nicht die Prozesse eines eShops nach. Stattdessen handelt es sich um eine CPQ-Lösung (Configure-Price-Quote). Das bedeutet: der Anwender wird durch den Konfigurationsprozess geführt, was in einer Preiskalkulation resultiert, woraus wiederum ein Angebot erstellt wird. Eine unmittelbare Bestellung mit Zahlungsabwicklung ist nicht vorgesehen. Der Anwendungsbereich liegt also im B2B \citep{tactonAbout}.

Technisch betrachtet ist TCsite eine Webanwendung. Im Lieferumfang sind 'Apache Tomcat' als Application Server sowie der \textbf{TCserver} enthalten. TCserver ist das, was gemäß Kapitel \ref{subssubsection:Produktkonfiguration} als eigentlicher Konfigurator bezeichnet wird. Er beherbergt die Konfigurationsengine. Damit ist das System gemeint, welches Konfigurationsaufgaben verarbeitet und Ergebnisse in unterschiedlicher Form präsentiert \citep{tactonTCsiteHandbook}. Abbildung \ref{fig:tcsiteHighLevel} veranschaulicht die High-Level-Architektur des Standardsetups.

\vspace{1em}
\begin{minipage}{\linewidth}
	\centering
	\includegraphics[width=1\linewidth]{Abbildungen/tcsiteHighLevel.pdf}
	\captionof{figure}[tcsiteHighLevel]{High-Level-Architektur von TCsite}
	\label{fig:tcsiteHighLevel}
\end{minipage}
\vspace{1em}

\subsubsection{Architektur}

Abbildung \ref{fig:tcsiteLowLevel} visualisiert die offene Schichtenarchitektur von TCsite. Offen bedeutet, dass jede Schicht mit allen darunter liegenden Schichten kommunizieren kann. Das Fundament bildet die Persistenzschicht. Sie realisiert die Datenhaltung. Die Plattform bildet ein \ac{API} mit den Basisfunktionalitäten für die darüber liegenden Schichten. Die Module realisieren jeweils eine der drei Oberflächenbereiche der Anwendung. Außerdem bieten sie Services und Erweiterungspunkte für die oberste Schicht: die Plugins. Durch diese kann die Funktionalität von TCsite individuell erweitert werden \citep{tactonTCsiteDevelopmentManual}.

\vspace{1em}
\begin{minipage}{\linewidth}
	\centering
	\includegraphics[width=1\linewidth]{Abbildungen/tcsiteLowLevel.pdf}
	\captionof{figure}[tcsiteLowLevel]{Schichtenarchitektur von TCsite}
	\label{fig:tcsiteLowLevel}
\end{minipage}
\vspace{1em}

Im Folgenden werden die Architekturkomponenten von unten nach oben vorgestellt.

\textbf{Repository}\\
Das Repository bietet eine Datenbank sowie ein Reihe von Funktionen zum Lesen und Schreiben der TCsite-Objekte. Das Repository ist in einen lokalen und einen globalen Speicher strukturiert. Wird ein Objekt erstellt oder geöffnet, geschieht die Bearbeitung immer auf einer Arbeitskopie in dem für jeden Nutzer spezifischen lokalen Speicher. Änderungen sind solange für andere Nutzer unsichtbar. Erst ein Commit sorgt für die nutzerübergreifende Sichtbarkeit im globalen Speicher. Bei dieser Übertragung wird eine Revisionshistorie über Objektänderungen geführt, so dass alte Zustande wieder abrufbar sind \citep{tactonTCsiteDevelopmentManual}.

\textbf{Administration}\\
Dieses Modul bildet die Administrationsoberfläche, dargestellt in Anhang \ref{app:tcsiteAdministration}. Hier werden Einstellungen getroffen und Verwaltungsaspekte realisiert. Dazu gehört zum Beispiel die Verwaltung der Nutzergruppen und Produkte.

Die Nutzergruppen definieren, welche Rechte ein TCsite-User hat. Standardmäßig verfügbar sind die Nutzergruppen \glqq Standarduser\grqq{}, \glqq Systemadministrator\grqq{} und \glqq Integrationuser\grqq{}. Letzterer Typ dient der Authentifizierung bei der Kommunikation mit externen Systemen, zum Beispiel bei einem Integrationsszenario. Bildlich personifiziert jede externe Webanfrage einen Gast, der die Maske des Integration Users aufgesetzt und dessen Rechte bekommt.

\vspace{1em}
\begin{minipage}{\linewidth}
	\centering
	\includegraphics[width=0.6\linewidth]{Abbildungen/tcsiteAdministrationProduct.PNG}
	\captionof{figure}[tcsiteAdministrationProduct]{Detailsansicht eines Produktes aus dem Produktkatalog}
	\label{fig:tcsiteAdministrationProduct}
\end{minipage}
\vspace{1em}

Außerdem wird hier der Produktkatalog verwaltet. Abbildung \ref{fig:tcsiteAdministrationProduct} zeigt die Detailsansicht eines Produktes. Die Darstellung macht deutlich, dass neben Produktbeschreibungen auch Dateien hinterlegt werden. Wird eine Modelldatei hochgeladen, ist das Produkt konfigurierbar. Ressourcen, wie z.B. Bilder, oder eine Übersetzungsdatei können ergänzend hinzugefügt werden \ref{tactonTCsiteReferenceManual}. Entsprechend der Definition eines Konfigurationsmodells in Kapitel \ref{subssubsection:Produktkonfiguration} wird so kein bestimmtes Produkt, sondern implizit alle seine Varianten angelegt.

\textbf{Quotation}\\
Das Quotation-Modul erstellt die Quotation-View, dargestellt in Abbildung \ref{fig:tcsiteQuotationNumbered}. Technisch betrachtet realisiert das Modul Methoden für die Manipulation und Anzeige von \textbf{Quotations} sowie zur Dokumentengenerierung aus der mit ihr im Zusammenhang stehenden Daten. Aus Anwendersicht ist eine Quotation ein Angebot, dass die für ihn konfigurierten Produkte enthält. Diese werden als \textbf{QuotationItems} bezeichnet und unter 'Products' gelistet (siehe [1]) \citep{tactonTCsiteDevelopmentManual}. Betrachtet man eine Quotation als Warenkorb, sind die QuotationItems die Warenkorbartikel. Der Vergleich hinkt insofern, als dass die Artikel eines Warenkorbs auch unabhängig von diesem existieren. Ein QuotationItem ist jedoch eine konkrete Variantenausprägung eines konfigurierbaren Produktes, die innerhalb der Quotation existiert.

\vspace{1em}
\begin{minipage}{\linewidth}
	\centering
	\includegraphics[width=0.7\linewidth]{Abbildungen/tcsiteQuotationNumbered.pdf}
	\captionof{figure}[tcsiteQuotationNumbered]{Quotation-View von TCsite}
	\label{fig:tcsiteQuotationNumbered}
\end{minipage}
\vspace{1em}


Die Produktkonfiguration wird über die Schaltfläche 'Add Product' gestartet (siehe [2]). Daraufhin wählt der Anwender das gewünschte Produkt aus einer Liste aus. Im Anschluss geschieht der Übergang zum Configuration-Modul \ref{tactonTCsiteReferenceManual}.

Eine Quotation hat weiterhin einen Lebenszyklus, welcher unter 'Lifecycle State' dargestellt wird (siehe [3]). Das Angebot befindet sich zu jedem Zeitpunkt in einem bestimmten Zustand, welche über den Administrationsbereich verwaltet werden. Die Zustandsabfolge definiert den sogenannten \glqq Workflow\grqq{}. Zustände unterscheiden sich in der Sichtbarkeit für verschiedene Nutzergruppen und deren Editierbarkeit. Beispielsweise kann sich eine Quotation in den Zuständen \glqq Design\grqq{} oder \glqq Offered\grqq{} befinden. In letzterem Zustand ist nicht mehr änderbar, damit keine Inkonsistenz zum herausgegebenen Angebot entstehenen kann \citep{tactonTCsiteReferenceManual}.

Die Box auf der rechten Seite stellt den Index aller Quotations dar (siehe [4]). Unter dem Suchfeld ist einstellbar, ob nur die Quotations des eingeloggten Nutzers gezeigt werden, oder ob nutzerübergreifend gelistet werden sollen. Wie im Administrationsbereich erwähnt, agieren auch externe Systeme als Nutzer, nämlich als Integration User. Auch Quotations, die von dieser Nutzergruppe stammen, werden hier gelistet  \citep{tactonTCsiteReferenceManual}.

\textbf{Configuration}\\
Das Configuration-Modul realisiert den C-Teil des CPQ-Prozesses: die Konfiguration. Es rendert die Konfigurationsansicht und verwaltet die mit der Konfiguration im Zusammenhang stehenden Objekte. Im Modul werden zwar  Konfigurationsaufgaben zusammengestellt, aber nicht gelöst. Dies übernimmt der TCserver. Somit ist die Konfiguration als Client-Server Architektur implementiert.

\vspace{1em}
\begin{minipage}{\linewidth}
	\centering
	\includegraphics[width=1\linewidth]{Abbildungen/tcsiteTCserverCommunication.pdf}
	\captionof{figure}[tcsiteTCserverCommunication]{Kommunikation zwischen TCsite und TCserver}
	\label{fig:tcsiteTCserverCommunication}
\end{minipage}
\vspace{1em}

Abbildung \label{fig:tcsiteTCserverCommunication} veranschaulicht die Phasen der Kommunikation zwischen TCsite und TCserver während der Produktkonfiguration. Der initiale Konfigurationsrequest dient dem Aufbau der Configuration-View entsprechend der Execution aus dem Konfigurationsmodell. Die Konfiguration startet immer in Step 1. Die Steps müssen in der vorgefertigten Reihenfolge abgearbeitet werden. Abbildung \ref{fig:tcsiteConfigurationNumbered} zeigt die Darstellung der Notebook-Konfiguration aus Kapitel \ref{subsubsection:Execution} während des zweiten Steps. Mittels [1] wird zwsichen den Steps gewechselt. [2] erlaubt die Navigation durch die Top-Level-Groups des aktuellen Steps. [3] stellt die Felder der aktuellen Top-Level-Group dar. Anhang \ref{app:tcSiteConfigurationOptionalGroups} zeigt anhand der 'Speicher'-Gruppe die Darstellung tieferer Group-Ebenen (siehe Kapitel \ref{subsubsection:Execution}). Außerdem wird bei [4] noch eine Top-Level-Group angezeigt: die sogenannte Info-Group. Allerdings wurde bei der Execution in Kapitel \ref{subsubsection:Execution} gar keine Info-Group explizit definiert. Offensichtlich wird eine Top-Level-Group dann als Info-Group dargestellt, wenn sie nur Felder vom Typ Label enthält.

\vspace{1em}
\begin{minipage}{\linewidth}
	\centering
	\includegraphics[width=0.8\linewidth]{Abbildungen/tcsiteConfigurationNumbered.pdf}
	\captionof{figure}[tcsiteConfigurationNumbered]{TCsite Configuration-View einer Notebook-Konfiguration}
	\label{fig:tcsiteConfigurationNumbered}
\end{minipage}
\vspace{1em}

Es fällt auf, dass bereits Optionen gewählt sind - obwohl der Anwender noch gar keine Wahl getroffen hat. Offensichtlich hat die Konfigurationsengine initial Default-Values gesetzt. Sie sind durch den grauen 'Please Confirm'-Hinweis gekennzeichnet.

\vspace{1em}
\begin{minipage}{\linewidth}
	\centering
	\includegraphics[width=0.8\linewidth]{Abbildungen/tcsiteConfigurationChoice.PNG}
	\captionof{figure}[tcsiteConfigurationChoice]{TCsite Configuration-View einer Notebook-Konfiguration nach Anwenderwahl}
	\label{fig:tcsiteConfigurationChoice}
\end{minipage}
\vspace{1em}

Nun beginnt die Loop-Phase, dargestellt in Abbildung \ref{fig:tcsiteTCserverCommunication}. Die Konfigurationsengine wird nach jeder Wahl des Anwenders aufgerufen, bis die Konfiguration im letzten Step beendet wird. Abbildung \ref{fig:tcsiteConfigurationChoice} zeigt die Konfiguration, nachdem der Anwender eine Option gewählt hat (Parameter: OS.type, Value: 'Windows CP'). Die gewählte Option wird unverhandelbar, gekennzeichnet durch das grüne 'Confirmed'. Außerdem wird überprüft, wie die gewählte Option die anderen verfügbaren Optionen beeinflusst. Values, die zu einem Konflikt führen, werden ausgegraut. Beim Feld 'Displaygröße' ist zu beobachten, dass die Wahl der Konfigurationsengine von '13' auf '15' geändert wurde. Offensichtlich sind Werte, die nicht vom Anwender gesetzt wurden, verhandelbar. Die Konfigurationsengine hält die Konfiguration folglich immer in einem korrekten Zustand. Durch einen Klick auf 'Confirm' bzw. 'Please Confirm'  können Optionen zwischen verhandelbar und unverhandelbar wechseln.

\vspace{1em}
\begin{minipage}{\linewidth}
	\centering
	\includegraphics[width=0.8\linewidth]{Abbildungen/tcsiteConfigurationConflict.PNG}
	\captionof{figure}[tcsiteConfigurationConflict]{TCsite Configuration-View in einer Konfliktsituation}
	\label{fig:tcsiteConfigurationConflict}
\end{minipage}
\vspace{1em}

Ausgegraute Optionen sind dennoch wählbar. Abbildung \ref{fig:tcsiteConfigurationConflict} zeigt die Configuration-View nach einem Klick auf die Displaygröße '13'. Dennoch ist '15' immer noch die aktive Option. Gleichzeitig wird am oberen Ende eine Konflikthinweis angezeigt. Die Konfigurationsengine weißt darauf hin, dass Optionen im Widerspruch stehen und schlägt eine Konfliktlösung vor. Diese kann angenommen oder abgelehnt werden. Im letzteren Fall bleibt die Konfiguration einfach in dem Zustand vor der Wahl der konfliktverursachenden Option. Auch die Eingabe von Werten in ein Textfeld kann zu Konflikten führen. Beispielsweise dann, wenn diese außerhalb der Domain des Parameters liegen oder im Widerspruch mit einer anderen Wahl stehen. 

Wird die Konfiguration über die Schaltfläche 'Ok' beendet, kehrt TCsite zum Quotation Modul zurück. Dort erscheint das konfigurierte Produkt in der Liste der QuotationItems.

\subsubsection{Erweiterbarkeit}
\label{subsubsection:Erweiterbarkeit}

Zur Erweiterung der Funktionalität bietet TCsite ein Pluginkonzept, welches Abbildung \ref{fig:tcsiteLowLevelExtensions} visualisiert. Neben den oben diskutierten Verantwortlichkeiten definiert die Plattform und die Module sogenannte Extensionpoints \citep{tactonTCsiteApiDocu}. Extensionpoints sind das, was in anderen Erweiterungskonzepten als Events bezeichnet wird: Ereignisse im Kontrollfluss der Geschäftsprozesse. Plugins können Methoden bereitstellen, die an den entsprechenden Ereignisstellen aufgerufen werden  \citep{tactonTCsiteDevelopmentManual}.

\vspace{1em}
\begin{minipage}{\linewidth}
	\centering
	\includegraphics[width=0.8\linewidth]{Abbildungen/tcsiteLowLevelExtensions.pdf}
	\captionof{figure}[tcsiteLowLevelExtensions]{TCsite Pluginkonzept}
	\label{fig:tcsiteLowLevelExtensions}
\end{minipage}
\vspace{1em}

Jedes Plugin hat Zugriff auf das Repository, die Plattform und das Moduls, zu dem es gehört. Die Plattform stellt außerdem ein paar spezielle Erweiterungspunkte bereit. Sie erlaubt unter anderem die Definition von Webcontrollern. Das bedeutet: Methoden, die auf HTTP-Requests an bestimmte URIs innerhalb des TCsite-Addressraums reagieren.

\textbf{Fazit}\\
In TCsite werden keine Varianten verwaltet, sondern Produkte mit Konfigurationsmodellen. Im Rahmen von Quotations werden über eine Konfiguration Variantenausprägungen generiert. Der Konfigurationsprozess ist im Configuration-Modul gekapselt. In diesem wird eine Nuzerschnittstelle gerendert. Die Verarbeitung der Konfigurationsaufgaben wird allerdings durch den TCserver realisiert. Dadurch ist die Implementierung einer Nutzerschnittstelle in einem anderen System denkbar, während der eigentliche Konfigurationsprozess nach wie vor in der Konfigurationsengine stattfindet. TCserver ist nicht unmittelbar ansprechbar. Das Pluginkonzept erlaubt jedoch die Definition neuer Serviceendpunkte über Webcontroller. An diesen können HTTP-Requests ausgewertet und beantworet werden. Das ermöglicht die Entwicklung eines Webservices, der eine Produktkonfiguration anbietet. Potentieller Client dieses Services ist jedes System, welches HTTP nutzen kann. Das trifft auf jede Webanwendung zu. eShops-Systeme sind per se Webanwendungen. Somit können technische Bedingungen an das eShop-System für das Integrationsszenario formuliert werden: (1) das System muss modifizierbar sein, (2) die Modifizierung muss einen Webservice in einer Client-Server Architektur nutzen können und (3) die Verwaltung von Varianten muss integraler Bestandteil des Systems sein.

??????????Jedes Shopsystem, welches diese technischen Bedingungen erfüllt, käme für das Integrationsszenario in Frage. Ein internen Evaluierungsprozess der Lino GmbH, bei dem zusätzlich ökonomische Faktoren bewertet wurden, hat zu Shopware als System für die prototypische Entwicklung geführt.??????

\subsection{Shopsystem}

Shopware wird von der shopware AG in Schöppingen entwickelt. Das eShop-System ist aus einer Individuallösung gewachsen, welche 2003 von den heutigen Geschäftsführern entwickelt wurde. Infolgedessen wurde das Unternehmen 2008 als Vertriebsgesellschaft für die Software gegründet. Aktuell liegt die fünfte Version vor. Shopware wird in einem Dual-License Modell vertrieben. Die Community Edition wird unter der \citeauthor{gnuAGPLv3} Open-Source Lizenz \glqq AGPLv3\grqq{} angeboten. Außerdem existieren drei kommerzielle Versionen \citep{shopwareUnternehmen}.

Im Folgenden wird Shopware einer technischen Analyse unterzogen. Darauf aufbauend wird die Erweiterbarkeit des Systems besprochen. Anschließend wird diskutiert, wie die Produktkonfiguration in der Standardinstallation technisch realisiert wird. Daraus kann eine Abgrenzung zum Tacton-Produktkonfigurator gewonnen und gleichzeitig dessen Integration motiviert werden. Die abschließende Betrachtung der Konfiguration im Einkaufsprozess liefert Ansatzpunkte für das Integrationskonzept.

\subsubsection{Architektur}
Shopware basiert auf der Programmiersprache PHP und verwendet eine relationale MySQL Datenbank. Grundlage des Technologiestacks ist das Zend-Framework, welches in der eigenentwickelten Abwandlung namens Enlight vorliegt \citep{shopware5Docs}.

Die Shopware-Architektur implementiert das \ac{MVC}-Muster, dargestellt in Abbildung \ref{fig:shopwareMVC}. Das \textbf{Model} definiert die Datenstrukturen des Systems (z.B. Artikel, Kategorien, Bestellungen etc.). Doctrine wird für die \ac{ORM} verwendet. Dadurch wird eine Abstraktionsschicht über die Datenbank aufgebaut \citep{shopware5Docs}. Das Framework ermöglicht die zentrale Definition der Datenbankstruktur in PHP und einen objektorientierten Datenzugriff \citep{shopware4Docs}.

\vspace{1em}
\begin{minipage}{\linewidth}
	\centering
	\includegraphics[width=1\linewidth]{Abbildungen/shopwareMVC.pdf}
	\captionof{figure}[shopwareMVC]{Shopware High-Level-Architektur}
	\label{fig:shopwareMVC}
\end{minipage}
\vspace{1em}

Die \textbf{View} nutzt die Templating-Engine Smarty. Diese erweitert den HTML-Syntax um spezielle Smarty-Tags. Sie ermöglichen unter anderem die Definition von Variablen zur Darstellung von Daten aus dem Model. Außerdem können über Smarty-Tags sogenannte Blöcke definiert werden. Es handelt sich hierbei um addressierbare Bereiche innerhalb eines Templates, die das Markup strukturieren. Sie spielen eine Rolle für das Erweiterungskonzept \citep{shopware5Docs}.

Controller sind Klassen, die Webrequests entgegen nehmen und eine Präsentation der Antwort durch Vermittlung zwischen Model und View entwickeln. Sie bieten Methoden an, welche als Actions bezeichnet werden. Diese sind die bestimmten URIs zugeordnet. Controller sind nach Verantwortungsbereichen aufgeteilt. Diese werden als Module bezeichnet und folgendermaßen typisiert \citep{shopware4Docs}:
\begin{compactitem}
\item \textbf{Frontend}-Controller sind für die Storefront zuständig. Das bedeutet: alle Seiten, die ein Kunde sieht.\\
Beispiele: Artikelliste einer bestimmten Kategorie, Detailseite eines Artikels, Warenkorb, Nutzeraccount, ...
\item \textbf{Widget}-Controller generieren wiederverwendbare Bestandteile der Storefront.\\
Beispiel: Liste der meistverkauften Artikel.
\item \textbf{Backend}-Controller sind für die Datenverwaltung der Shopadministration zuständig. Sie generieren jedoch keine Einzelseiten wie im Falle der Frontend-Controller. Stattdessen ist der administrative Bereich als Singe-Page-Application mittels des Javascript-Frameworks Ext JS implementiert. Dieses stellt eine Menge an Steuerelementen (z.B. Menüs, Formulare, etc.) zur Verfügung.\\
Beispiele: Nutzerverwaltung, Artikelverwaltung, Pluginverwaltung, ...
\item Für externe Systeme, die mit den Ressourcen von Shopware interagieren möchten, ist eine \textbf{REST-API} implementiert. Ein vollständige Übersicht aller Ressourcenendpunkte ist \citep{shopwareRestApiEndpunkte} zu entnehmen.\\
Beispiel: Abrufen des Artikels mit der ID 167.
\end{compactitem}

Der Großteil der Logik ist jedoch nicht in den Actions, sondern den sogenannten core-Klassen implementiert \citep{shopware4Docs}. Beispielsweise wird der Request zum Hinzufügen eines Warenkorbartikels vom Checkout-Controller entgegen genommen, die notwendigen Geschäftsprozesse finden aber in der Serviceklasse 'sBasket' statt.

\subsubsection{Erweiterbarkeit}
Der Quellcode der Community-Edition liegt offen. Theoretisch ist eine Erweiterung der Funktionalität über ein direktes Eingreifen in den Shopware-Kern denkbar. Der Hersteller sieht jedoch eine Anpassung über Plugins vor. Entsprechend der \ac{MVC}-Architektur wird in logische (Controller), Daten- (Model) und Template- (View) Erweiterungen unterschieden \citep{shopware5Docs}.

\textbf{Logische Erweiterungen}\\
Logische Erweiterungen werden über Events und Hooks realisiert. Events sind \glqq definierte Ereignisse, die im Workflow des Shops auftreten\grqq{} \citep{shopware4Docs}. Plugins können Code registrieren, welcher an den Ereignispunkten ausgeführt wird. Steht kein Event für die geplante Modifikation zur Verfügung, kann Plugin-Code über das Hooksystem unmittelbar auf Funktionen des Shopware-Kerns registriert werden\citep{shopware5Docs}. Damit ist das Erweiterungskonzept flexibler als das von TCsite.

Events werden in Controller-Events und Notify-Events unterschieden \citep{shopware4Docs}. \textbf{Controller-Events} sind an den Dispatch-Vorgang gekoppelt. Der shopware-Entwickler \citet{noegel15Diaspatch} definiert Dispatching als einen Prozess, bei dem das Resquest-Objekt gehandhabt, daraus das relevante Modul, der Controller und die Action extrahiert, der entsprechende Controller instanziiert und dieser zur Behandlung des Request gebracht wird. Leitet ein Controller den Request zu einem anderen Controller weiter, wiederholt sich dieser Vorgang. Plugins können auszuführenden Code registrieren, der vor (\textbf{PreDispatch}) oder nach dem Dispatching (\textbf{PostDispatch}) ausgeführt werden soll. Außerdem können sie den Dispatch-Prozess auf eigene Funktionen umleiten und so ganze Controller-Methoden ersetzen. \textbf{Notify-Events} entsprechen hingegen den in Kapitel \ref{subsubsection:Erweiterbarkeit} erwähnten Extension-Points von TCsite. Sie finden beispielsweise in den Core-Klassen Verwendung. So kann abseits der Controller in den Programmablauf eingegriffen werden \citep{shopware4Docs}.

Hooks bieten einen generischeren Ansatz. Events sind auf den Dispatchprozess und alle sonstigen Punkte beschränkt, an denen ein Shopware-Entwickler Erweiterbarkeit vorgedacht hat. Das Hooksystem bezeichnet die Möglichkeit, jede public und protected-Funktion von Shopware zu Erweitern. Sie erlauben die Modifizierung der Eingangsparameter und Rückgabewerte der Originalfunktion sowie deren komplette Ersetzung \citet{noegel15Hooks}.

\textbf{Daten-Erweiterungen}\\
Im Gegensatz zu TCsite erlaubt Shopware die Erstellung und Modifizierung von Datenbanktabellen. Sollen bestehende Datenmodelle nur um bestimmte Eigenschaften ergänzt werden sollen, existiert das Attributsystem. Gewisse Shopware-Entitäten (z.B. s\_user für Shopkunden) haben Attributtabellen (z.B. s\_user\_attributes) in einer 1-zu-1 Beziehung zugeordnet. Plugins können diesen beliebige Spalten hinzufügen (z.B. die Lieblingsfarbe des Kunden) \citep{shopware5Docs}.

\textbf{Template-Erweiterungen}\\
Die View bietet Erweiterungspunkte über das Smarty-Block-System. Die Blöcke können durch eigenen Templatecode ersetzt oder erweitert werden. Weiterführend sind so eigene CSS- und Javascriptdateien im Seitenkopf einbindbar. Folglich kann durch Template-Erweiterungen auch clientseitige Logik implementiert werden.

\subsubsection{Konfiguration}
Shopware behauptet in der der offiziellen Funktionsübersicht, bereits in der Standardinstallation einen Konfigurator zu besitzen \citep{shopware5Funktionsuebersicht}. Im Folgenden wird dessen Implementierung analysiert. Daraufhin wird dessen Integration in den Einkaufsvorgangs diskutiert.

\textbf{Technische Analyse}\\
Ein Artikel wird über den entsprechenden Menüpunkt im Backend angelegt (siehe Abbildung \ref{fig:shopwareBackendArtikel}). Über das Userinterface kann dieser als \glqq Varianten-Artikel\grqq{} gekennzeichnet werden. Dadurch wird der Variantenreiter aktiviert, unter dem der Administrator alle Varianten des Artikels festlegt.

Dazu werden sogenannte \glqq Gruppen\grqq{} (z.B. 'Storage') angelegt, die wiederum \glqq Optionen\grqq{} (z.B. 'HDD' oder 'SSD') besitzen. Das ist äquivalent zu den Begriffen \glqq Parameter\grqq{} und \glqq Domain\grqq{} in der Execution des Tacton-Konfigurationsmodells, welches in Kapitel \ref{subsubsection:Execution}) vorgestellt wurde. Es werden also die Wahlmöglichkeiten des Anwenders definiert, jedoch ohne Konfigurationswissen. Die Gruppen und Optionen werden separat in der Datenbank gespeichert und sind Artikelübergreifend verwendbar. Über den Button \glqq Varianten generieren\grqq{} (siehe Abbildung 	\ref{app:shopwareBackendArtikelVarianten}) werden alle Varianten explizit in der Datenbank angelegt, die sich aus der Kombination der Optionen ergeben. Das Notebook-Beispiel aus Kapitel \ref{subsection:Konfigurationsmodell} würde damit folgende Rechnung ergeben:

$3_{Massenspeicher1}*3_{Massenspeicher2}*2_{Arbeitsspeicher}*2_{Displaygr"o"se}*2_{Betriebssystem}*4_{Anzahl CPU-Kerne}*99_{Anzahl iTunes-Music-Pakete} = 28512  Varianten$

Damit passiert das, was durch das Konfigurationsmodell laut Kapitel \ref{subsubsection:begriffsuberblick} verhindert werden soll: das explizite definieren und abspeichern aller Varianten. Es gibt keine Constraints. Dementsprechend muss der Administrator händisch alle inkorrekte Varianten löschen und und die Preise der korrekten Varianten eintragen. Die in Kapitel \ref{subsubsection:begriffsuberblick} vorgestellte Konfigurationsdefinition \citet{sabin98} fordert aber Constraints. Somit stellt Shopware keine Konfiguration bereit, sondern die Selektion einer explizit definierten Variante entsprechend der gewählten Optionen aus einer Datenbank.

\textbf{Einkaufsvorgang}\\
Ausgehend von dieser Analyse ist der Einkaufsvorgang eines Varianten-Artikels unter Berücksichtigung der technischen Prozesse als Aktivitätsdiagramm in Abbildung \ref{fig:shopwareKonfigurationFlussdiagramm} darstellbar .

\vspace{1em}
\begin{minipage}{\linewidth}
	\centering
	\includegraphics[width=1\linewidth]{Abbildungen/shopwareKonfigurationFlussdiagramm.pdf}
	\captionof{figure}[shopwareKonfigurationFlussdiagramm]{Fussdiagramm der Einkaufsvorgangs eines Varianten-Artikels in Shopware}
	\label{fig:shopwareKonfigurationFlussdiagramm}
\end{minipage}
\vspace{1em}

Mittels der Katalog- oder Suchfunktion wählt der Kunde einen konfigurierbaren Artikel aus. So kommt er zu dessen Detailseite. Neben beschreibenden Texten und Bildern befindet sich hier die Konfigurationsoberfläche (siehe Abbildung 
\ref{app:shopwareNotebookDetail}). Zu jeder Gruppe wird ein Menü mit den entsprechenden Optionen angezeigt. Die Auswahl einer Option führt zum Reload der Seite mit der aktuellen Variante. Dieser Vorgang wiederholt sich so lange, bis der Artikel den Vorstellungen des Kunden entspricht. Außerdem wird die Anzahl eingestellt. Über den \glqq In den Warenkorb\grqq{} Button wird der Artikel in den Warenkorb gelegt. Ist diese Variante bereits im Warenkorb, wird die Anzahl des entsprechenden Warenkorbartikels inkrementiert. Anderenfalls entsteht eine neue Warenkorbposition. Im Warenkorb ist an der jeweiligen Artikelbezeichnung ablesbar, welche Optionskombination vorliegt (siehe Abbildung \ref{app:shopwareNotebookWarenkorb}). Nun überprüft der Kunde den Warenkorb. Möchte er weitere Artikel hinzufügen, beginnt der Prozess mit der Auswahl des Wunschartikels von vorne. Ist der Kunde mit einer Warenkorbposition noch nicht zufrieden, kommt er über einen Klick auf den entsprechenden Artikel zur Detailsansicht dieser Variante zurück. Nun können andere Optionen gewählt werden. Ein erneuter klick auf \glqq In den Warenkorb\grqq{} aktualisiert nicht etwa die zugehörige Warenkorbposition, sondern erzeugt einen neue. Die alte Variante verbleibt zusätzlich im Warenkorb. Ist der Nutzer mit alle Warenkorbartikeln zufrieden, wird der Bestellvorgang über den Checkout-Prozess abgeschlossen.

\subsection{Fazit}
Zwei Plugins entwickeln.
Tacton bietet Configuration als Service an.
Showpare nutzt den Service.
Shopware verlagert die Produktkomplexität damit aus dem eigenen System. Außerdem sind ATO-Produkte anbietbar.

\pagebreak

\section{Anforderungen}

\subsection{Funktionale Anforderungen}

\subsection{Nichtfunktionale Anforderungen}

\pagebreak

\section{Integrationskonzept}

\pagebreak

\section{Integrationsumsetzung}

\pagebreak

\section{Fazit}

\pagebreak

% ----------------------------------------------------------------------------------------------------------
% Vorlagen
% ----------------------------------------------------------------------------------------------------------
\section{Vorlagen}
Dieses Kapitel enthält Beispiele zum Einfügen von Abbildungen, Tabellen, etc.

\subsection{Bilder}
Zum Einfügen eines Bildes, siehe Abbildung \ref{fig:osgi}, wird die \textit{minipage}-Umgebung genutzt, da die Bilder so gut positioniert werden können.

\vspace{1em}
\begin{minipage}{\linewidth}
	\centering
	\includegraphics[width=0.7\linewidth]{Bilder/layering-osgi.png}
	\captionof{figure}[OSGi Architektur]{OSGi Architektur\footnotemark }
	\label{fig:osgi}
\end{minipage}
\footnotetext{Quelle: \url{http://www.osgi.org/Technology/WhatIsOSGi}}

\subsection{Tabellen}
In diesem Abschnitt wird eine Tabelle (siehe Tabelle \ref{tab:beispiel}) dargestellt.

\vspace{1em}
\begin{table}[!h]
	\centering
	\begin{tabular}{|l|l|l|}
		\hline
		\textbf{Name} & \textbf{Name} & \textbf{Name}\\
		\hline
		1 & 2 & 3\\
		\hline
		4 & 5 & 6\\
		\hline
		7 & 8 & 9\\
		\hline
	\end{tabular}
	\caption{Beispieltabelle}
	\label{tab:beispiel}
\end{table}

\pagebreak
\subsection{Auflistung}
Für Auflistungen wird die \textit{compactitem}-Umgebung genutzt, wodurch der Zeilenabstand zwischen den Punkten verringert wird.

\begin{compactitem}
	\item Nur
	\item ein
	\item Beispiel.
\end{compactitem}

\subsection{Listings}
Zuletzt ein Beispiel für ein Listing, in dem Quellcode eingebunden werden kann, siehe Listing \ref{lst:arduino}.

\vspace{1em}
\begin{lstlisting}[caption=Arduino Beispielprogramm, label=lst:arduino]
int ledPin = 13;
void setup() {
    pinMode(ledPin, OUTPUT);
}
void loop() {
    digitalWrite(ledPin, HIGH);
    delay(500);
    digitalWrite(ledPin, LOW);
    delay(500);
}
\end{lstlisting}

\subsection{Tipps}
Die Quellen befinden sich in der Datei \textit{bibo.bib}. Ein Buch- und eine Online-Quelle sind beispielhaft eingefügt.

Abkürzungen lassen sich natürlich auch nutzen. Weiter oben im Latex-Code findet sich das Verzeichnis.
\pagebreak

% ----------------------------------------------------------------------------------------------------------
% Literatur
% ----------------------------------------------------------------------------------------------------------
\pagenumbering{Roman}
\setcounter{page}{1}

\renewcommand\refname{Quellenverzeichnis}
\lhead{}
\bibliographystyle{myalpha}
\bibliography{bibo}
\pagebreak

% ----------------------------------------------------------------------------------------------------------
% Anhang
% ----------------------------------------------------------------------------------------------------------
\pagenumbering{Roman}
\setcounter{page}{1}
\lhead{Anhang \thesection}

\begin{appendix}
\section*{Anhang}
\phantomsection
\addcontentsline{toc}{section}{Anhang}
\addtocontents{toc}{\vspace{-0.5em}}

\vspace{1em}
\begin{minipage}{\linewidth}
	\centering
	\includegraphics[width=1\linewidth]{Abbildungen/tcsiteAdministration.PNG}
	\captionof{figure}[tcsiteAdministration]{TCsite Administrationsoberfläche}
	\label{app:tcsiteAdministration}
\end{minipage}
\vspace{1em}

\vspace{1em}
\begin{minipage}{\linewidth}
	\centering
	\includegraphics[width=1\linewidth]{Abbildungen/tcSiteConfigurationOptionalGroups.PNG}
	\captionof{figure}[tcSiteConfigurationOptionalGroups]{Darstellung optionaler Groups in TCsite}
	\label{app:tcSiteConfigurationOptionalGroups}
\end{minipage}
\vspace{1em}

\end{appendix}

\chapter*{Selbstständigkeitserklärung}
\vspace{2em}
Ich versichere hiermit an Eides statt, dass ich die vorliegende Bachelorarbeit selbstständig und ohne unzulässige fremde Hilfe erbracht habe. Ich habe keine anderen als die angegebenen Quellen und Hilfsmittel benutzt sowie wörtliche und sinngemäße Zitate kenntlich gemacht. Die Arbeit hat in gleicher oder ähnlicher Form noch keiner Prüfungsbehörde
vorgelegen.

\vspace{4em}
\begin{minipage}{\linewidth}
	\begin{tabular}{p{15em}p{15em}}
		Datum: &  .......................................................\\
		& \centering (Unterschrift)\\
	\end{tabular}
\end{minipage}

\end{document}
