% ----------------------------------------------------------------------------------------------------------
% Verzeichnisse
% ----------------------------------------------------------------------------------------------------------
% TODO Typ vor Nummer
\renewcommand{\cfttabpresnum}{Tab. }
\renewcommand{\cftfigpresnum}{Abb. }
\settowidth{\cfttabnumwidth}{Abb. 10\quad}
\settowidth{\cftfignumwidth}{Abb. 10\quad}

\titlespacing{\section}{0pt}{12pt plus 4pt minus 2pt}{2pt plus 2pt minus 2pt}
\singlespacing
\phantomsection

\addtocounter{section}{1}
\rhead{Inhaltsverzeichnis}
\tableofcontents
\pagebreak
\rhead{Verzeichnisse}
\listoffigures
\pagebreak
\listoftables
\pagebreak
\renewcommand{\lstlistlistingname}{Listing-Verzeichnis}
{\labelsep2cm\lstlistoflistings}
\pagebreak

% ----------------------------------------------------------------------------------------------------------
% Abkürzungen
% ----------------------------------------------------------------------------------------------------------
\section*{Abkürzungsverzeichnis}
\begin{acronym}[WSDL] % längste Abkürzung steht in eckigen Klammern
	\setlength{\itemsep}{-\parsep} % geringerer Zeilenabstand
	\acro{API} {application programming interface}
	\acro{ATO}{Assemble-to-Order}
	\acro{B2B}{Business-to-Business}
	\acro{B2C}{Business-to-Customer}
	\acro{CSP}{Constraint Satisfaction Problem}
	\acro{ETO}{Engineer-to-Order}
	\acro{MC}{Mass Customization}
	\acro{MTO}{Make-to-Order}
	\acro{PTO}{Pick-to-Order}
	\acro{RPC}{Remote Procedure Call}
	\acro{WSDL} {Web Service Description Language}
\end{acronym}
\newpage