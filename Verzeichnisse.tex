% ----------------------------------------------------------------------------------------------------------
% Verzeichnisse
% ----------------------------------------------------------------------------------------------------------
% TODO Typ vor Nummer


\rhead{Inhaltsverzeichnis}
\setcounter{secnumdepth}{3} 
\setcounter{tocdepth}{3}
\tableofcontents
\pagebreak
\rhead{Verzeichnisse}
% ----------------------------------------------------------------------------------------------------------
% Abkürzungen
% ----------------------------------------------------------------------------------------------------------
\chapter*{Abkürzungsverzeichnis}
\addcontentsline{toc}{chapter}{Abkürzungsverzeichnis} 
\begin{acronym}[KE-Output] % längste Abkürzung steht in eckigen Klammern
	\setlength{\itemsep}{-\parsep} % geringerer Zeilenabstand
	\acro{AJAX}{asynchronous JavaScript and XML}
	\acro{API}{Application Programming Interface}
	\acro{ATO}{Assemble-to-Order}
	\acro{B2B}{Business-to-Business}
	\acro{B2C}{Business-to-Customer}
	\acro{CORS}{Cross-Origin Resource Sharing}
	\acro{CSP}{Constraint Satisfaction Problem}
	\acro{d.h.}{das heißt}
	\acro{etc.}{et cetera}
	\acro{ETO}{Engineer-to-Order}
	\acro{i.d.R.}{in der Regel}
	\acro{JSON}{JavaScript Object Notation}
	\acro{KE-Input}{Konfigurationsengine-Input}
	\acro{KE-Output}{Konfigurationsengine-Output}
	\acro{MC}{Mass Customization}
	\acro{MTO}{Make-to-Order}
	\acro{MVC}{Model View Controller}
	\acro{ORM}{objektrelationale Abbildung}
	\acro{PTO}{Pick-to-Order}
	\acro{RPC}{Remote Procedure Call}
	\acro{SOP}{Same-Origin-Policy}
	\acro{UML}{Unified Modeling Language}
	\acro{URI}{Uniform Resource Identifier}
	\acro{vgl.}{vergleiche}
	\acro{WSDL}{Web Service Description Language}
	\acro{z.B.}{zum Beispiel}
\end{acronym}
\pagebreak
\listoffigures
\pagebreak
\listoftables
\pagebreak
\lstlistoflistings
\pagebreak